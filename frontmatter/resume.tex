\documentclass[../master.tex]{subfiles}
\begin{document}
\section*{Resume}
I dette projekt har vi studeret området for data-flow analyse for programmeringssproget ReScript.
ReScript er et funktionelt programmeringssproget baseret på OCaml, med fokus by at compile til JavaScript.
I forhold til andre sprog med at compile til JavaScript som mål, introducerer ReScript et robust type system.
Selvom ReScript er baseret på OCaml, så anvender ReScript deres eget build system og kun anvender dele af OCaml.
Dette, og at ReScript stadigvæk er et ungt sprog, mangler ReScript analyse værktøjer der kan hjælpe udvikler og optimerer compileren.
De har dog et eksperimentelt analyseværktøj, ReAnalyze, som kan give informationer såsom død-code og terminering.

Her er data-flow analyse en teknik brugt til at samle information der bliver brugt igennem et program, som ofte anvendes til at optimere et program.
Under optimeringsprocessen, bliver data-flow analysen brugt til en række forskellige områder, såsom eliminering af død-kode og konstant propagation.
Eftersom mange programmeringssproget introducer lokationer og pointers som en del af sproget, skal data-flow analysen også tage højde for aliasering, altså hvilke variabler der deler samme location, som bliver brugt til at sørger for sikkerheden af analysen.
\bigskip

Til dette har vi valgt at lave et data-flow analyse værktøj for ReScript, hvor vi har valgt at fokuserer på en delmængde af ReScript sproget, altså for et $\lambda$-calculus med mønstermatching, muterbarhed, og lokale deklarationer.
Eftersom vi analyserer på et funktionelt sprog, som derved er udtryk baseret, laver vi analysen på forekomster af syntaktiske og semantiske elementer, altså af udtryk, variabler, og lokationer.
Siden vi er intereseret i forekomster, annoterer vi forekomster med program punkter.

Hertil, præsenterer vi en formel beskrivelse af sproget, dets syntaks og semantik, hvor det præsenteret semantiske formål er at vise den semantiske data-flow et program.
For at kunne repræsentere data-flow for et program indsamler vi de semantiske forekomster anvendt til at evaluere en forekomst.
Vi definerer afhængigheder af, som de variabler og lokationer, anvendt i en evaluering.
Vi introducerer også en afhængighedsfunktioner som binder deklaration, argumenter fra abstraktioner, og lokation forekomster til de forekomster de er afhængige af.

Siden afhængighedsfunktionen ikke indeholder en ordning i sig selv, bruger vi også en relation af de programpunkter, der er i afhængighedsfunktionen.
Denne relation bliver anvendt når der skal slås op i afhængighedsfunktionen, e.g., når vi skal slå op hvad en variabel eller lokation er afhængig af, hvortil vi anvender en funktion til at slå op for det maksimale programpunkt en forekomst er bundet til.
\bigskip

Baseret på sproget, definerer vi et type system for data-flow analyse, med formål at samle afhængigheder der bliver brugt og alias information for en syntaktisk forekomst.
Siden de egenskaber vi vil fange i et program er forekomst afhængighed og aliasing, præsenterer vi type som repræsenterer de forskellige værdier i sproget, altså funktioner, rekursive funktioner, lokationer, etc.
Siden lokationer er en semantisk notation, introduceres notation for interne variable, som er den syntaktiske repræsentation af lokationer.

Vi introducerer en base type og abstraktions-type, hvor abstraktions-typen repræsenterer funktioner, mens base typen repræsenterer typen af de andre værdier.
Hertil består basen af en mængde af forekomster og mængden af alias informationen.
Mængden af forekomster bliver brugt til at samle de forekomster der er blevet anvendt til at evaluerer en forekomst, og dermed den syntaktiske repræsentation og afhængighed.
Alias information er en mængde, der kan indeholde variabler og intern variable.
Denne mængde repræsenterer, hvis værdien af en forekomst er en lokation, så vil alias mængden indeholde information omkring lokationen.

Ligesom semantikken, introducerer vi type miljøet, som er en approksimation af afhængighedsfunktionen.
Hvortil vi, ligesom to afhængighedsfunktion, introducerer en funktion til at slå op for det maksimale element.
Siden type miljøet også indeholder global information, anvender vi en funktion til at slå op for alle maksimale forekomster.
Siden vi approksimerer i type system, kan der være grene, introduceret af mønstermatchingen, hvor vi sørger for at få den maksimale forekomst for hver gren.

Til sidst præsenterer vi sundhedsresultatet af type systemet, hvor vi viser hvordan bindingsmodellerne i type systemet og semantikken relaterer til hinanden, og at typen respekterer afhængighederne i semantikken.

\newpage
\end{document}
