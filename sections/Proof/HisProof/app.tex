\item[\runa{App}] Here $e^{p'}=[e_1^{p_1}\;e_2^{p_2}]^{p'}$, where
\begin{figure}[H]
	\setlength\tabcolsep{8pt}
	\begin{tabular}{l}
		\runa{App}\\[0.2cm]
	\inference[]
	{
		env \vdash \left\langle e_1^{p_1},sto,(w,\sqsubseteq_w),p \right\rangle \rightarrow \left\langle v,sto',(w_1,\sqsubseteq_w^1),(L_1,V_1),p_1 \right\rangle &\\
		env \vdash \left\langle e_2^{p_2},sto',(w_1,\sqsubseteq_w^1),p_1 \right\rangle \rightarrow \left\langle v',sto'',(w_2,\sqsubseteq_w^2),(L_2,V_2),p_2 \right\rangle &\\
		env'[x\mapsto v'] \vdash \left\langle e_3^{p_3},sto'',(w_3,\sqsubseteq_w^3),p_2 \right\rangle \rightarrow \left\langle v'',sto',(w',\sqsubseteq_w'),(L_3,V'_3),p_3 \right\rangle
	}
	{env\vdash \left\langle \left[e_1^{p_1}\;e_2^{p_2}\right]^{p'},sto,(w,\sqsubseteq_w),p \right\rangle \rightarrow \left\langle v'',sto',(w',\sqsubseteq_w'),(L,V),p' \right\rangle}\\[0.3cm]
	Where $(L,V)=(L_1\cup L_3,V_1\cup V_3)$, $v=\left\langle x,f,e_3^{p_3},env'\right\rangle$, $w_3=w_2[x^{p_2}\mapsto (L_2,V_2)]$

	\end{tabular}
\end{figure}
By virtue of our induction hypothesis, we can get the follow from the premises:
\begin{description}
	\item[1)] if $y^{p''}\in dom(w_1)\backslash dom(w)$ then $y\notin fv(e_1^{p_1})$
	\item[2)] if $y^{p''}\in dom(w_2)\backslash dom(w_1)$ then $y\notin fv(e_2^{p_2})$
	\item[3)] if $y^{p''}\in dom(w')\backslash dom(w_3)$ then $y\notin fv(e_3^{p_3})$
\end{description}
We the need to show that: if $y^{p''}\in dom(w')\backslash dom(w)$ then $y\notin fv(e^{p'})$.
By \cref{def:fv} we know that $fv([e_1^{p_1}\;e_2^{p_2}]^{p'})=fv(e_1^{p_1})\cup fv(e_2^{p_2})$, and the only variable that is not handled by \cat{1)}, \cat{2)}, and \cat{3)} is $x^{p_2}$.
But since $x$ is not a free variable in $e_1^{p_1}$ or $e_2^{p_2}$, this case then follows.
