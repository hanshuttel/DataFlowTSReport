\item[\runa{App}] Here $e^{p'}=\left[e_1^{p'}\;e_2^{p''}\right]^{p'}$, where
\begin{figure}[H]
	\setlength\tabcolsep{8pt}
	\begin{tabular}{l}
		\runa{App}\\[0.2cm]
	\inference[]
	{
		env \vdash \left\langle e_1^{p_1},sto,(w,\sqsubseteq_w),p \right\rangle \rightarrow \left\langle v,sto',(w_1,\sqsubseteq_w^1),(L_1,V_1),p_1 \right\rangle &\\
		env \vdash \left\langle e_2^{p_2},sto',(w_1,\sqsubseteq_w^1),p_1 \right\rangle \rightarrow \left\langle v',sto'',(w_2,\sqsubseteq_w^2),(L_2,V_2),p_2 \right\rangle &\\
		env'[x\mapsto v'] \vdash \left\langle e_3^{p_3},sto'',(w_3,\sqsubseteq_w^3),p_2 \right\rangle \rightarrow \left\langle v'',sto',(w',\sqsubseteq_w'),(L_3,V'_3),p_3 \right\rangle
	}
	{env\vdash \left\langle \left[e_1^{p_1}\;e_2^{p_2}\right]^{p'},sto,(w,\sqsubseteq_w),p \right\rangle \rightarrow \left\langle v'',sto',(w',\sqsubseteq_w'),(L,V),p' \right\rangle}\\[0.3cm]
	Where $(L,V)=(L_1\cup L_3,V_1\cup V_3)$, $v=\left\langle x,f,e_3^{p_3},env'\right\rangle$, $w_3=w_2[x^{p_2}\mapsto (L_2,V_2)]$

	\end{tabular}
\end{figure}

And from our assumptions, we have that:
\begin{itemize}
	\item $\Gamma,\Pi\vdash \left[e_1^{p_1}\;e_2^{p_2}\right]^{p'}:T$,
	\item $\Gamma,\Pi\vdash env$
	\item $(env,sto,(w,\sqsubseteq_w))\models(\Gamma,\Pi)$,
\end{itemize}
To type $[e_1^{p_1}\;e_2^{p_2}]^{p'}$ we need to use the \runa{T-App} type rule, where we have:
\begin{figure}[H]
	\setlength\tabcolsep{8pt}
	\begin{tabular}{l}
		\runa{T-App}\\[0.2cm]
			\inference[]
			{\Gamma,\Pi\vdash e_1^{p_1}:T_1\rightarrow T &\\
			\Gamma,\Pi\vdash e_2^{p_2}:T_1}
			{\Gamma,\Pi\vdash [e_1^{p_1} \; e_2^{p_2}]^{p'}:T}
	\end{tabular}
\end{figure}
We must show that \cat{1)} $\Gamma,\Pi\vdash v:T$, \cat{2)} $(env,sto',(w',\sqsubseteq_w'))\models(\Gamma,\Pi)$, and \\
\cat{3)} $(env,v,(w',\sqsubseteq_w'),(L,V))\models(\Gamma,T)$.

\begin{description}
	\item[1)] To conclude, we first need to look at the premises.
	Due to the first premise: since from our assumptions we have $(env,sto,(w,\sqsubseteq_w))\models(\Gamma,\Pi)$ and $\Gamma,\Pi\models env$, and from the first premise of \runa{T-App}, we can get the following our induction hypothesis:
	\begin{itemize}
		\item $\Gamma,\Pi\vdash v_1:T_1\rightarrow T$,
		\item $(env,sto_1,(w_1,\sqsubseteq_w^1))\models(\Gamma,\Pi)$,
		\item $(env,v,(w_1,\sqsubseteq_w^1),(L_1,V_1))\models(\Gamma,T_1\rightarrow T)$
	\end{itemize}

	Due to the first premise and the second premise of \runa{T-App} we can get the following our induction hypothesis:
	\begin{itemize}
		\item $\Gamma,\Pi\vdash v_2:T_1$,
		\item $(env,sto_2,(w_2,\sqsubseteq_w^2))\models(\Gamma,\Pi)$,
		\item $(env,v,(w_2,\sqsubseteq_w^2),(L_2,V_2))\models(\Gamma,T_1)$
	\end{itemize}

	For the third premise, we first need to look at the value of the first premise, where since we know it is an abstraction, it is concluded by the \runa{Closure} rules:
	\begin{figure}[H]
		\setlength\tabcolsep{8pt}
		\begin{tabular}{l}
			\runa{Closure}\\[0.4cm]
				\inference[]
				{
					\Gamma,\Pi\vdash env' \\
					\Gamma_1,\Pi\vdash e_3^{p_3}:T
				}
				{\Gamma;\Pi\vdash \left\langle x^{p_0}, e_3^{p_3}, env' \right\rangle:T_1\rightarrow T}
		\end{tabular}
	\end{figure}
	Where $\Gamma_1=\Gamma[x^{p_0}:T_1]$.
	We then need to show that there exists a $\Gamma_1$ such that 
	\cat{a)} $\Gamma_1;\Pi\vdash e_3^{p_3}:T$, and 
	\cat{b)} $(env'[x\mapsto v_2],sto_2,(w_3,\sqsubseteq_w^3))\models(\Gamma_1,\Pi)$.
	\begin{description}
		\item[a)] From our assumption, we know that $\Gamma,\Pi\vdash env$ and since we do not allow binding a closure to a location, we then also know that $env=env'[env'']$ for some $env''$, i.e., $env$ contains more bindings than $env'$.
			Based on this, we then also know that $\Gamma,\Pi\vdash env'$, which then conclude the first premise of the \runa{Closure} rule.
	
			For the second premise, we need to show that there exists a $x^{p_0}:T_1$, where from the \runa{T-Abs} rule we know that $p_0\sqsubseteq_\Pi p_3$.
			Since we know that $env[x\mapsto v_2]$, and from the second premise of \runa{T-App} that $\Gamma;\Pi\vdash v_2:T_1$ and $w_2'=w_2[x^{p_2}\mapsto(L_2,V_2)]$, there must be a $x^{p_2}$ such that $p_0=p_2$ and $\Gamma_1=\Gamma[x^{p_2}:T_1]$.

		\item[b)] We know that $(env,sto_2,(w_2,\sqsubseteq_w^2))\models(\Gamma,\Pi)$ from the second premise.
			From \cat{a)} we know that $env=env'[env'']$ as such we also know that $(env',sto_2,(w_2,\sqsubseteq_w^2))\models(\Gamma,\Pi)$.
			We then need to show that $(env'[x\mapsto v_2],sto_2,(w_3,\sqsubseteq_w^2))\models(\Gamma_1,\Pi)$, where $w_3=w_2[x^{p_2}\mapsto(L_2,V_2)]$ and $\Gamma_1=\Gamma[x^{p_2}:T_1]$.
		
			We then only need to show that the updates for $x^{p_2}$ holds.
			Since $x^{p_2}$ is bound in $w_3$, $\Gamma_1$, and $x$ is bound in $env$, that $(env,v,(w_2,\sqsubseteq_w^2),(L_2,V_2))\models(\Gamma,T_1)$, and that $p_2 \sqsubseteq_w^2 p_2$.
			By \cref{def:EnvAgree} we then know that $(env'[x\mapsto v_2],sto_2,(w_3,\sqsubseteq_w^2))\models(\Gamma_1,\Pi)$.
	\end{description}
	From \cat{a)} and \cat{b)}, we can then conclude the third premise of \runa{App} by our induction hypothesis:
	\begin{itemize}
		\item $\Gamma_1,\Pi\vdash v:T$,
		\item $(env'[x\mapsto v_2],sto',(w',\sqsubseteq_w'))\models(\Gamma_1,\Pi)$,
		\item $(env'[x\mapsto v_2],(w',\sqsubseteq_w'),(L_3,V_3))\models(\Gamma,T)$
	\end{itemize}
	As such, since $x^{p_2}\notin fv(v)$ then by \cref{lemma:Strength} we get
	$\Gamma,\Pi\vdash v:T$

\item[2)] Due to \cat{1)} we know that $(env'[x\mapsto v_2],sto',(w',\sqsubseteq_w'))\models(\Gamma_1,\Pi)$, and from \cref{lemma:Strength} we know $x^{p_2}\notin fv(v)$.
	From \cref{lemma:His} we can get the following:
	\begin{itemize}
		\item if $x^{p''}\in dom(w_1)\backslash dom(w)$ then $x^{p''}\notin fv(e_1^{p_1})$
		\item if $x^{p''}\in dom(w_2)\backslash dom(w_1)$ then $x^{p''}\notin fv(e_2^{p_2})$
		\item if $x^{p''}\in dom(w')\backslash dom(w_3)$ then $x^{p''}\notin fv(e_3^{p_3})$
	\end{itemize}
	We then also know that: if $x^{p''}\in dom(w')\backslash dom(w)$ then $x^{p''}\notin fv(e^{p'})$.
	We can then get $(env',sto',(w',\sqsubseteq_w'))\models(\Gamma,\Pi)$.
	Since $env=env'[env'']$ and we know that $(env,sto',(w_3,\sqsubseteq_w^2))\models(\Gamma,\Pi)$, due to \cref{def:EnvAgree} since $w_3=w_2[x^{p_2}\mapsto(L_2,V_2)]$ and $x^{p_2}\notin dom(\Gamma)$.
	We can then get:
	$$(env,sto',(w',\sqsubseteq_w'))\models(\Gamma,\Pi)$$

\item[3)] Due to \cat{1)} and \cat{2)} we can then immediatly conclude that:
	$$(env,v,(w',\sqsubseteq_w'),(L,V))\models(\Gamma,T)$$
\end{description}
