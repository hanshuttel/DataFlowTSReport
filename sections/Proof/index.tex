\documentclass[../../master.tex]{subfiles}
\begin{document}
\section{Soundness}\label{sec:Soundness}
We will now show the soundness of the type system, i.e., the type of an occurrence correspond to the dependencies and the alias information from the semantics.
To show that the type system is sound, we will first introduce the type rules for values and the relation between the semantics and the type system.
After that, we will present some properties in the semantics and type system that are used in the soundness proof.
And lastly, we will show the soundness of the type system.

\subfile{valueTRules.tex}
\subfile{agreement.tex}

\subsection{Properties}
Before we present the soundness proof, we will first present some properties about the semantics and type system.
These properties presents some areas about local information, and as such uses \cref{def:fv}.

The first property we present is a property of the dependency function.
Since the dependency function is global, and as such they can contain side effects after evaluation, 
any new variable bindings introduced to the dependency function by evaluating an occurrence $e^p$, those variables are then not free in $e^p$.

\begin{lemma}[History]\label{lemma:His}
	Suppose $e^p$ is an occurrence, that
	$$env\vdash\left\langle e^{p},sto,(w,\sqsubseteq_w),p'\right\rangle\rightarrow\left\langle v,sto',(w',\sqsubseteq_w'),(L,V),p''\right\rangle$$
		and $x^{p_1}\in dom(w')\backslash dom(w)$.
		Then $x\notin fv(e^{p})$
\end{lemma}
The proof for \cref{lemma:His} can be found in \cref{app:HisProof}.

%\subfile{HisProof/index.tex}

The second property is the strengthening of the type environment.
This strengthening lemma states that if there is a binding the type environment, used to type an occurrence $e^p$, and the variables is not free in $e^p$ then the binding can be removed.

\begin{lemma}[Strengthening]\label{lemma:Strength}
	If $\Gamma[x^{p'}:T'],\Pi\vdash e^{p}:T$ and $x\notin fv(e^p)$, then $\Gamma,\Pi\vdash e^{p}:T$
\end{lemma}
The proof for \cref{lemma:Strength} can be found in \cref{app:StrProof}.
%\subfile{StrProof/index.tex}

With history, \cref{lemma:His}, and strengthening, \cref{lemma:Strength}, defined we can then show the soundness theorem.
This theorem compares the semantics for an occurrence to a type rule that concludes for this occurrence.
Since we are interested if the type system is a sound approximation of the semantics, we need to make sure that an evaluation of an occurrence and the type for the occurrence agrees.
As such, we assume that we the type environment and approximated order of program points are in an agreement with the binding models in the semantics.
We also assume that all bindings in the environment are well-typed.

Based on this we then need to make sure that, after an evaluation, that environments are still in agreement, and that the type we get is in agreement with the semantics.
Lastly, we also need to make sure that we can type the value that an evaluation step produces.

\begin{theorem}[Soundness of type system]
	Suppose $e^{p'}$ is an occurrence where
	\begin{itemize}
		\item $env\vdash\left\langle e^{p'},sto,(w,\sqsubseteq_w),p\right\rangle\rightarrow\left\langle v,sto',(w',\sqsubseteq_w'),(L,V),p''\right\rangle$,
		\item $\Gamma,\Pi\vdash e^{p'} : T$
		\item $\Gamma,\Pi\vdash env$
		\item $(env,sto,(w,\sqsubseteq_w))\models(\Gamma,\Pi)$
	\end{itemize}
	Then we have that:
	\begin{itemize}
		\item $\Gamma,\Pi\vdash v:T$
		\item $(env,sto',(w',\sqsubseteq_w'))\models(\Gamma,\Pi)$
		\item $(env,(w',\sqsubseteq_w'),v,(L,V))\models(\Gamma,T)$
	\end{itemize}
\end{theorem}
\begin{proof}
	The proof proceeds by induction on the height of the derivation tree for 
	$$env\vdash\left\langle e^{p'},sto,\psi,p\right\rangle\rightarrow\left\langle v,sto',\psi',(L,V),p''\right\rangle$$
	We will only show the proof of four rules here, for \runa{Var}, \runa{Case}, \runa{Ref}, and \runa{Ref-write}, the full proof can be found in \cref{app:SoundnessProof}.

	\begin{description}
		\item[\runa{Var}] Here $e^{p'}=x^{p'}$, where
\begin{figure}[H]
	\setlength\tabcolsep{8pt}
	\begin{tabular}{l}
		\runa{Var}\\[0.2cm]
	\inference[]{}
	{env\vdash \left\langle x^{p'},sto,(w,\sqsubseteq_w),p \right\rangle \rightarrow \left\langle v,sto,(w,\sqsubseteq_w),(L,V\cup\{x^{p'}\}),p' \right\rangle}\\[0.3cm]
	Where $env(x)=v$, $x^{p''}=uf_{\sqsubseteq_w}(x,w)$, and $w(x^{p''})=(L,V)$

	\end{tabular}
\end{figure}
And from our assumptions, we have:
\begin{itemize}
	\item $\Gamma,\Pi\vdash x^{p'} : T$
	\item $\Gamma,\Pi\vdash env$
	\item $(env,sto,(w,\sqsubseteq_w))\models(\Gamma,\Pi)$
\end{itemize}
To type the occurrence $x^{p'}$ we use the rule \runa{T-Var}:
\begin{figure}[H]
	\setlength\tabcolsep{8pt}
	\begin{tabular}{l}
		\runa{T-Var}\\[0.2cm]
			\inference[]{}
			{\Gamma,\Pi \vdash x^p:T \sqcup (\{x^p\},\emptyset)}
	\end{tabular}
\end{figure}
Where $x^{p''}=uf_{\sqsubseteq_\Pi}(x,\Gamma)$, $\Gamma(x^{p''})=T$.

We need to show that \cat{1)} $\Gamma,\Pi\vdash c:T$, \cat{2)} $(env,sto',(w',\sqsubseteq_w'))\models(\Gamma,\Pi)$, and \cat{3)} $(env,v,(w',\sqsubseteq_w'),(L,V))\models(\Gamma,T)$.
\begin{description}
	\item[1)] Since, from our assumption, we know that $\Gamma,\Pi\vdash env$, we can then conclude that $\Gamma,\Pi\vdash v:T$

	\item[2)] Since there are no updates to $sto$, $(w,\sqsubseteq_w)$, we then know from our assumptions that $(env,sto,(w,\sqsubseteq_w))\models(\Gamma,\Pi)$ holds after an evaluation.

	\item[3)] Since there are no updates to $sto'$, $(w',\sqsubseteq_w')$, that $(L,V)$ is a result for looking up $x^{p''}$ in $(w,\sqsubseteq_w)$, and the type $T$ is due to looking up $x^{p''}$ in $\Gamma$, we then know that $(env,v,(w',\sqsubseteq_w'),(L,V))\models(\Gamma,T)$.
		We then need to show for $(env,v,(w',\sqsubseteq_w'),(L,V\cup\{x^{p''}\}))\models(\Gamma,T\sqcup \{x^{p''}\})$ which holds due to \cref{def:TAgree}.
\end{description}

		\item[\runa{Case}] Here $e^{p'}=\left[\mbox{case}\;e^{p''}\;\tilde{\pi}\;\tilde{o}\right]^{p'}$, where
\begin{figure}[H]
	\setlength\tabcolsep{8pt}
	\begin{tabular}{l}
		\runa{Case}\\[0.2cm]
	\inference[]
	{
		env \vdash \left\langle e^{p''},sto,(w,\sqsubseteq_w),p \right\rangle \rightarrow \left\langle v_e,sto'',(w'',\sqsubseteq_w''),(L'',V''),p'' \right\rangle &\\
		env[env'] \vdash \left\langle e_j^{p_j},sto'',(w''',\sqsubseteq_w''),p'' \right\rangle \rightarrow \left\langle v,sto',(w',\sqsubseteq_w'),(L',V'),p_i \right\rangle
	}
	{env\vdash \left\langle \left[\mbox{case}\;e^{p''}\;\tilde{\pi}\;\tilde{o}\right]^{p'},sto,(w,\sqsubseteq_w),p \right\rangle \rightarrow \left\langle v,sto',(w',\sqsubseteq_w'),(L,V),p' \right\rangle}\\[0.3cm]
	Where $match(v_e,s_i)=\perp$ for all $1\leq u<j\leq|\tilde{\pi}|$, $match(v_e,s_j)=env'$, and \\
	$w'''=w''[x\mapsto(L'',V'')]$ if $env'=[x\mapsto v_e]$ else $w'''=w''$

	\end{tabular}
\end{figure}

And from our assumptions, we have that:
\begin{itemize}
	\item $\Gamma,\Pi\vdash \left[\mbox{case}\;e^{p''}\;\tilde{\pi}\;\tilde{o}\right]^{p'}:T$,
	\item $\Gamma,\Pi\vdash env$
	\item $(env,sto,(w,\sqsubseteq_w))\models(\Gamma,\Pi)$,
\end{itemize}
To type $[\mbox{case}\;e^{p''}\;\tilde{\pi}\;\tilde{o}]^{p'}$ we need to use the \runa{T-Case} rule, where we have:
\begin{figure}[H]
	\setlength\tabcolsep{8pt}
	\begin{tabular}{l}
		\runa{T-Case}\\[0.2cm]
			\inference[]
				{\Gamma,\Pi\vdash e^{p}:(\delta,\kappa) &\\
				\Gamma',\Pi\vdash e_i^{p_i}:T_i\;\;\;(1\leq i\leq|\tilde{\pi}|)}
				{\Gamma,\Pi\vdash [\mbox{case}\;e^{p}\;\tilde{\pi}\;\tilde{o}]^{p'}:T}
	\end{tabular}
\end{figure}
Where $T=T'\sqcup(\delta,\kappa)$, $T'=\bigcup_{i=1}^{|\tilde{\pi}|}T_i$, $e_i^{p_i}\in\tilde{o}$ and $s_i\in\tilde{\pi}$, and $\Gamma'=\Gamma[x^p:(\delta,\kappa)]$ if $s_i=x$.

We must show that \cat{1)} $\Gamma,\Pi\vdash v:T$, \cat{2)} $(env,sto',(w',\sqsubseteq_w'))\models(\Gamma,\Pi)$, and \\
\cat{3)} $(env,v,(w',\sqsubseteq_w'),(L,V))\models(\Gamma,T)$.

To conclude, we first need to show for the premise, where due to our assumption and from the first premise, we can use the induction hypothesis, where we then get:
\begin{itemize}
	\item $\Gamma,\Pi\vdash v_e:(\delta,\kappa)$,
	\item $(env,sto'',(w'',\sqsubseteq_w''))\models(\Gamma,\Pi)$,
	\item $(env,v,(w'',\sqsubseteq_w''),(L,V))\models(\Gamma,(\delta,\kappa))$
\end{itemize}
Since in the rule \runa{T-Case} we take the union of all patterns, we can then from the second premise:
\begin{itemize}
	\item $\Gamma,\Pi\vdash v:T_j$,
	\item $(env,sto',(w',\sqsubseteq_w'))\models(\Gamma,\Pi)$,
	\item $(env,v,(w',\sqsubseteq_w'),(L,V))\models(\Gamma,T_j)$
\end{itemize}

If we have \cat{a)} $\Gamma',\Pi\vdash env[env']$ and \cat{b)} $(env[env'],sto'',(w''',\sqsubset_w''))\models(\Gamma',\Pi)$, we can then conclude by our induction hypothesis.
\begin{description}
	\item[a)] We know that either we have $\Gamma'=\Gamma[x\mapsto(\delta,\kappa)]$ and $env[x\mapsto v_e]$ if $s_j=x$, or $\Gamma'=\Gamma$ and $env$ if $s_j\neq x$.
		\begin{itemize}
			\item if $s_j\neq x$: Then we just have $\Gamma,\Pi\vdash env$
			\item if $s_j=x$: Then we have $\Gamma[x\mapsto(\delta,\kappa)],\Pi\vdash env[x\mapsto v_e]$, which hold due to to the first premise.
		\end{itemize}
	\item[b)] We know that either we have $\Gamma'=\Gamma[x\mapsto(\delta,\kappa)]$ and $env[x\mapsto v_e]$ if $s_j=x$, or $\Gamma'=\Gamma$ and $env$ if $s_j\neq x$.
		\begin{itemize}
			\item if $s_j\neq x$: then we have $(env,sto'',(w'',\sqsubset_w''))\models(\Gamma,\Pi)$.
			\item if $s_j=x$: then $(env[x\mapsto v_e],sto'',(w''',\sqsubset_w''))\models(\Gamma[x\mapsto(\delta,\kappa)],\Pi)$, since we know that $(env,sto'',(w'',\sqsubset_w''))\models(\Gamma,\Pi)$, we only need to show for $x$.
				Since we have $x\in dom(env)$ and $x^{p_j}\in dom(w''')$ and $x^{p_j}\in dom(\Gamma')$ and due to the 1 premise, we then know that $(env[x\mapsto v_e],sto'',(w''',\sqsubset_w''))\models(\Gamma[x\mapsto(\delta,\kappa)],\Pi)$.
		\end{itemize}
\end{description}
Based on \cat{a)} and \cat{b)} we can then conclude:

\begin{description}
	\item[1)] Since $\Gamma',\Pi\vdash v:T_j$, then we also must have $\Gamma',\Pi\vdash v:T$, since $T$ only contains more information than $T_j$.
	\item[2)] By the second premise, \cref{lemma:His}, and \cref{lemma:Strength}, we can then get $(env,sto',(w',\sqsubseteq_w'))\models(\Gamma,\Pi)$.
	\item[3)] Due to \cat{1)}, \cat{2)}, \cat{a)}, and \cat{b)} we can then conclude that $(env,v,(w',\sqsubseteq_w'),(L,V))\models(\Gamma,T)$.
\end{description}

		\item[\runa{ref}] Here $e^{p'}=\left[\mbox{ref}\;e_1^{p_1}\right]^{p'}$, where
\begin{figure}[H]
	\setlength\tabcolsep{8pt}
	\begin{tabular}{l}
		\runa{Ref}\\[0.2cm]
	\inference[]
	{env \vdash \left\langle e^{p'},sto,(w,\sqsubseteq_w),p \right\rangle \rightarrow \left\langle v,sto',(w',\sqsubseteq_w'),(L,V),p' \right\rangle}
	{env\vdash \left\langle \left[\mbox{ref}\;e^{p'}\right]^{p''},sto,(w,\sqsubseteq_w),p \right\rangle \rightarrow \left\langle \loc,sto'',(w'',\sqsubseteq_w'),(\emptyset,\emptyset),p'' \right\rangle}\\[0.3cm]
	Where $\loc=next$, $sto''=sto'[next\mapsto new(\loc),\loc\mapsto v]$, and\\
	$w''=w'[\loc^{p'}\mapsto (L,V)]$

	\end{tabular}
\end{figure}
From our assumption, we know that 
\begin{itemize}
	\item $\Gamma,\Pi\vdash \left[\mbox{ref}\;e_1^{p_1}\right]^{p'}:T$,
	\item $\Gamma,\Pi\vdash env$
	\item $(env,sto,(w,\sqsubseteq_w))\models(\Gamma,\Pi)$,
\end{itemize}
To type $[\mbox{ref}\;e_1^{p_1}]^{p'}$ we need to use the \runa{T-Ref} type rule:
\begin{figure}[H]
	\setlength\tabcolsep{8pt}
	\begin{tabular}{l}
		\runa{T-Ref}\\[0.2cm]
			\inference[]
				{\Gamma,\Pi\vdash  e_1^{p_1}:(\delta',\kappa')}
				{\Gamma[\nu x^{p'}:(\delta',\kappa')],\Pi\vdash [\mbox{ref}\;e_1^{p_1}]^{p'}:(\emptyset,\kappa)}
	\end{tabular}
\end{figure}
Where $\kappa=\{\nu x\}$.
We need to show that \cat{1)} $\Gamma,\Pi\vdash [\lambda\;x.e^{p''}]:T$, \cat{2)} $(env,sto',(w',\sqsubseteq_w'))\models(\Gamma,\Pi)$, and \cat{3)} $(env,v,(w',\sqsubseteq_w'),v,(L,V))\models(\Gamma,T)$.

To conclude, we first need to show for the premise, where due to our assumption and from the premise, we can use the induction hypothesis, where we then get:
\begin{itemize}
	\item $\Gamma,\Pi\vdash v_1:(\delta',\kappa')$,
	\item $(env,sto'',(w'',\sqsubseteq_w'))\models(\Gamma,\Pi)$,
	\item $(env,v,(w'',\sqsubseteq_w'),(L,V))\models(\Gamma,(\delta',\kappa'))$
\end{itemize}
We are also going to denote $\Gamma'=\Gamma[\nu x^{p'}:(\delta',\kappa')]$.

\begin{description}
	\item[1)] Since we know that \runa{Ref} evaluates to a location, we know it should be concluded by \runa{Closure}.
	\begin{figure}[H]
		\setlength\tabcolsep{8pt}
		\begin{tabular}{l}
			\runa{Location}\\[0.2cm]
				\inference[]{}
					{\Gamma,\Pi\vdash  \loc:(\delta, \kappa)}
		\end{tabular}
	\end{figure}
	Where $\kappa\neq\emptyset$.
	From the \runa{T-Ref} rule, we know that the type is $(\emptyset,\{\nu x\})$, we can then conclude that $\delta=\emptyset$ and $\kappa=\{\nu x\}$.
	\item[2)] Since we know that $(env,sto'',(w'',\sqsubseteq_w'))\models(\Gamma,\Pi)$, we then need to show for the extension to $sto''$, $w''$, and $\Gamma'$.
		Due to \cat{1)}, $(env,v,(w'',\sqsubseteq_w'),(L,V))\models(\Gamma,(\delta',\kappa'))$, and since we bind $\loc^{p'}$ in $sto''$ and $\nu x^{p'}$ in $\Gamma$, we can then conclude that $(env,sto',(w',\sqsubseteq_w'))\models(\Gamma',\Pi)$
	\item[3)] Due to \cat{1)} and \cat{2)} we can conclude that $(env,v,(w',\sqsubseteq_w'),(L,V))\models(\Gamma,T)$.
\end{description}

		\item[\runa{Ref-read}] Here $e^{p'}=[!e_1^{p_1}]^{p'}$, where
\begin{figure}[H]
	\setlength\tabcolsep{8pt}
	\begin{tabular}{l}
		\runa{Ref-read}\\[0.2cm]
			\inference[]
				{env \vdash \left\langle e_1^{p_1},sto,(w,\sqsubseteq_w),p \right\rangle \rightarrow \left\langle \loc,sto',(w',\sqsubseteq_w'),(L,V),p_1 \right\rangle}
				{env\vdash \left\langle \left[!e_1^{p_1}\right]^{p'},sto,(w,\sqsubseteq_w),p \right\rangle \rightarrow \left\langle v,sto',(w',\sqsubseteq_w'),(L\cup\{\loc^{p''}\},V),p' \right\rangle}
	\end{tabular}
\end{figure}
Where $sto'(\loc)=v$.
And from our assumptions, we have that:
\begin{itemize}
	\item $\Gamma,\Pi\vdash [!e_1^{p_1}]^{p'}:T$,
	\item $\Gamma;\Pi\vdash env$
	\item $(env,sto,(w,\sqsubseteq_w))\models(\Gamma,\Pi)$,
\end{itemize}
To type $[!e_1^{p_1}]^{p'}$ we need to use the \runa{T-Ref-read} rule, where we have:
\begin{figure}[H]
	\setlength\tabcolsep{8pt}
	\begin{tabular}{l}
		\runa{T-Ref-read}\\[0.2cm]
			\inference[]
				{\Gamma,\Pi\vdash  e^{p}:(\delta,\kappa)}
				{\Gamma,\Pi\vdash [!e^{p}]^{p'}:T\sqcup(\delta\cup\delta',\emptyset)}\\
	\end{tabular}
\end{figure}
Where $\kappa\neq\emptyset$, $\delta'=\{\nu x^{p'}\mid\nu x\in\kappa\}$, $\nu x_1,\cdots,\nu x_n\in\kappa$.\\ 
$\{\nu x_1^{p_1},\cdots,\nu x_1^{p_m}\}=uf_{\Upsilon_{p'}}(\nu x_1,\Gamma),\cdots,\{\nu x_n^{p_1'},\cdots,\nu x_n^{p_s'}\}=uf_{\Upsilon_{p'}}(\nu x_n,\Gamma)$, and\\
$T=\Gamma(\nu x_1^{p_1})\cup\cdots\cup\Gamma(\nu x_1^{p_m})\cup\cdots\cup\Gamma(\nu x_n^{p_1'})\cup\cdots\cup\Gamma(\nu x_n^{p_s'})$.

We must show that \cat{(1)} $\Gamma,\Pi\vdash v:T$, \cat{(2)} $(env,sto',(w',\sqsubseteq_w'))\models(\Gamma,\Pi)$, and \cat{(3)} $(env,(w',\sqsubseteq_w'),(L,V))\models(\Gamma,T)$.

To conclude, we first need to show for the premise, where due to our assumption and from the premise, we can use the induction hypothesis, where we then get:
\begin{itemize}
	\item $\Gamma,\Pi\vdash \loc:(\delta,\kappa)$,
	\item $(env,sto',(w',\sqsubseteq_w'))\models(\Gamma,\Pi)$,
	\item $(env,(w',\sqsubseteq_w'),(L,V))\models(\Gamma,(\delta',\kappa'))$
\end{itemize}

Due to $(env,sto',(w',\sqsubseteq_w'))\models(\Gamma,\Pi)$ and $(env,(w',\sqsubseteq_w'),(L,V))\models(\Gamma,(\delta',\kappa'))$, and due to our assumption we can conclude that:
\begin{description}
	\item[(1)] $\Gamma,\Pi\vdash v:T$,

	\item[(2)] $(env,sto',(w',\sqsubseteq_w'))\models(\Gamma,\Pi)$,

	\item[(3)] $(env,(w',\sqsubseteq_w'),(L\cup\{\loc^{p''}\},V))\models(\Gamma,T\sqcup(\delta\cup\delta',\emptyset))$
\end{description}

	\end{description}
\end{proof}

%\subfile{SoundProof/index.tex}
\end{document}
