\documentclass[../../master.tex]{subfiles}
\begin{document}
\subsection{Type rules for values}
Before introducing the soundness theorem, we will first introduce the notion of typing values.
Many of these rules can be directly derived from the type system, while we need to introduce some new rules.

The type rules for values are, shown in \cref{fig:ValTypeRules}, as follows:

\begin{figure}[H]
	\setlength\tabcolsep{8pt}
	\begin{tabular}{l}
		\hline\\
		\runa{Constant}\\[0.2cm]
			\inference[]{}
				{\Gamma;\Pi\vdash  c:(\emptyset, \emptyset)}\\[1cm]

		\runa{Location}\\[0.2cm]
			\inference[]{}
				{\Gamma;\Pi\vdash  \loc:(\delta, \kappa)}\\[1cm]

		\runa{Closure}\\[0.2cm]
			\inference[]
				{
					\Gamma;\Pi\vdash env \;\;\;
					\Gamma,x^{p}:T_1;\Pi\vdash e^{p'}:T_2
				}
				{\Gamma;\Pi\vdash \left\langle x^{p}, e^{p'}, env \right\rangle^{p''}:T_1\rightarrow T_2}\\[1cm]

				\todo[inline]{The \runa{Recursive closure} type rule is pretty shady}\\
		\runa{Recursive closure}\\[0.2cm]
			\inference[]
				{
					\Gamma;\Pi\vdash env \;\;\;
					\Gamma,x^{p}:T_1;\Pi\vdash e^{p''}:T_2
				}
				{\Gamma;\Pi\vdash \left\langle x^{p}, f^{p'}, e^{p''}, env \right\rangle^{p_3}:T_1\rightarrow T_2}\\[1cm]

		\runa{Unit}\\[0.2cm]
			\inference[]{}
			{\Gamma;\Pi\vdash  ():(\delta, \kappa)}\\[0.5cm]
		\hline
	\end{tabular}
	\caption{Type rules for values}
	\label{fig:ValTypeRules}
\end{figure}

\paragraph{Constants}
The first type rule for constants, \runa{Constant}, is nearly identical to \runa{Const}, where the only difference is that \runa{Const} is for a constant occurrence, and \runa{Constant} is for a constant.

\paragraph{Locations}
The location type rule, \runa{Location}, is given to a base type, where $\kappa\neq\emptyset$ as there must be exist a corresponding internal variable.

\todo[inline]{The \runa{Closure} rule should be discussed further, mostly in regards to the typing of $env$}
\paragraph{Closure}
The closure type rule, \runa{Closure}, types a closure with the function type.
This rule is similar to the \runa{Abs} rule, where the input type is $T_1$ and the body produces the type $T_2$.
In addition, it also types the environment $env$, where this typing is an agreement between the variable occurrences in $\Gamma$ and variables in $env$.

\paragraph{Recursive closure}
ss

\paragraph{Unit}
The unit type rule, \runa{Unit}, only appears in the \runa{Ref read} rule, when writing to a reference.
This rule retains the dependencies used to evaluate the left-hand side of the write rule.

\end{document}
