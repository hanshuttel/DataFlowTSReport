\documentclass[../../master.tex]{subfiles}
\begin{document}
\subsection{Type rules for values}
For the sake of proving the type system, we present type rules for the values presented int \cref{sec:EnvSto}, where the type rules is given in \cref{fig:ValTypeRules}.

As the values for closures and recursive closures contains an environment, from where they where declared, as such, before introducing the type rules for values we will present the notion for well-typed environments.

\begin{definition}[Environment judgement]\label{def:TEnv}
	Let $v_1,\cdots,v_n$ be values such that $\Gamma,\Pi\vdash v_i:T_i$, for $1\leq i\leq n$.
	Let $env$ be an environment given by $env=[x_1\mapsto v_1,\cdots,x_n\mapsto v_n]$, $\Gamma$ be a type environment, and $\Pi$ be the approximated order of program points.
	We say that:
	$$\Gamma,\Pi\vdash env$$
	iff 
	\begin{itemize}
		\item For all $x_i\in dom(env)$ then $\exists x_i^p\in dom(\Gamma)$ where $\Gamma(x_i^p)=T_i$ then 
			$$\Gamma,\Pi\vdash env(x_i):T_i$$
	\end{itemize}
\end{definition}

In \cref{def:TEnv} we show the notion of well-typed environments, which states that: given the type of all values, $T_i$, for all variables, $x_i$, bound in the $env$, 
then there exists an occurrence of $x$ in $\Gamma$, where the type from looking up for that occurrences is is $T_i$.
As such, we know that all bindings $x$ in $env$ also have a type in $\Gamma$ from when $x$ were declared.

\begin{description}
	\item[\runa{Constant}] type rule differs from the rule \runa{T-Const}, since most occurrences can evaluate to a constant and as such we know that its type should be a base type.
		Since constants can depend on other occurrences, we know that $\delta$ can be non-empty, but since constants are not locations, we also know that it cannot contain alias information, and as such $\kappa$ should be empty.

	\item[\runa{Location}] type rule represents locations, where we know that it must be a base type.
		Since locations can depend on other occurrences, we know that $\delta$ can be non-empty.
		As locations can contains alias information, and that a location is considered to always be an alias to itself, we know that $\kappa$ can never be empty, as it should always contain an internal variable. 

	\item[\runa{Closure}] type rule represents abstraction, and as such we know that it should have the abstraction type, $T_1\rightarrow T_2$, where we need to make an assumption about the argument type $T_1$.
		Since a closure contains the parameter, body, and the environment for an abstraction from when it were declared, we also need to handle those part in the type rule.
		The components of the closure is handled in the premises, where the environment must be well-typed.
		We also type the body of the abstraction, where we know that we need to update the type environment with the type $T_1$ for its parameter, Where we type the body with $T_2$.

	\item[\runa{Recursive closure}] type rules is similar to the \runa{Closure} rule, but since this is a recursive closure, we additionally need to update the type environment with the name of the recursive binding to the type of the abstraction.

	\item[\runa{Unit}] type rule simply have the base type, as it is not an abstraction and it also cannot have alias information.
		As the unit value is introduced from writing to references, $o=[o_1\;:=\;o_2]^p$, we know that from the type rule \runa{Ref-write} that the dependencies from the occurrence $o$ should also contain the set of occurrences.
		As such, the \runa{Unit} rule also contains a set of occurrences, $\delta$.
\end{description}

\begin{figure}[H]
	\setlength\tabcolsep{8pt}
	\begin{tabular}{l}
		\runa{Constant}\\[0.2cm]
			\inference[]{}
				{\Gamma,\Pi\vdash  c:(\delta, \emptyset)}\\[1cm]

		\runa{Location}\\[0.2cm]
			\inference[]{}
				{\Gamma,\Pi\vdash  \loc:(\delta, \kappa)}\\
				Where $\kappa\neq\emptyset$\\[1cm]

		\runa{Closure}\\[0.2cm]
			\inference[]
				{
					\Gamma,\Pi\vdash env \\
					\Gamma[x^{p}:T_1],\Pi\vdash e^{p'}:T_2
				}
				{\Gamma,\Pi\vdash \left\langle x^{p}, e^{p'}, env \right\rangle^{p''}:T_1\rightarrow T_2}\\[1cm]

		\runa{Recursive closure}\\[0.2cm]
			\inference[]
				{
					\Gamma,\Pi\vdash env \\
					\Gamma[x^{p}:T_1,f^{p'}:T_1\rightarrow T_2],\Pi\vdash e^{p''}:T_2
				}
				{\Gamma,\Pi\vdash \left\langle x^{p}, f^{p'}, e^{p''}, env \right\rangle^{p_3}:T_1\rightarrow T_2}\\[1cm]

		\runa{Unit}\\[0.2cm]
			\inference[]{}
				{\Gamma,\Pi\vdash  ():(\delta,\emptyset)}\\[0.5cm]
	\end{tabular}
	\caption{Type rules for values}
	\label{fig:ValTypeRules}
\end{figure}
\end{document}
