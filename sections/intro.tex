\documentclass[../../master.tex]{subfiles}
\begin{document}
\section{Introduction}
Data-flow analysis has been studied for decades to better to provide flow information of programs.
This flow information has been used for different tasks for compiler optimization, debugging and understanding programs, testing and maintenance.
In the context of compiler optimization, where the flow information provides data that may be used at given parts of the program at runtime.

One of the classical way of doing data-flow analysis has been iterative algorithm using graph representation of programs.
The purpose of those graphs is to provide the control flow of a program by providing the control flow as edges and the nodes as basic blocks.
By using the information of control-flow graphs, many algorithms have been developed to use those by annotating the graphs and solving the maximal fix-point\cite{KildallGaryA1973Auat}. (Ref to algorithms, such as kilders)
Other techniques have also been presented, such as a graph-free approach \cite{HorspoolR.Niegel2002AGAt} or through type systems with refinement types \cite{PavlinovicZvonimir2021Dfrt}.

When analysing languages, such as C/C++ or other languages that explicitly handles pointers, it is important to take into account aliasing, i.e., multiple variables referring to the same location.
Many data-flow analysis uses alias algorithms to compute this information.
Two overall types of alias algorithm has been used, flow-sensitive which give precise information but are expensive, and flow-insensitive which are less precise but are inexpensive \cite{LiangDonglin1999Eaag}(ref to alias techniques).
\bigskip

This paper will focus on presenting a type system based approach for data-flow analysis for functional languages.
Functional languages have seen a rise in popularity, as they take a different approach to programming languages as they take an expression based approach.
In essence, functional languages that are based on $\lambda$-calculus are in essence a series function declarations and applications of those functions.

Many different functional languages are being used and adopted, as many of those languages often have a strong type systems.
Functional languages are also beginning to rise for web application, where ReScript is an alternative to TypeScript.
The ReScript programming language is based on OCaml, with a JavaScript inspired syntax \cite{rescript_rebrand}, and compared to languages such as JavaScript and TypeScript, ReScript is a strongly typed language.
\todo[inline]{Add something more about rescript, analysis, etc.}

We will then present a type system for data-flow analysis for a ReScript inspired $\lambda$-calculus language with mutability, bindings, and pattern matching.
The type system we proposes provides the data information used at each program point and the alias information used.
\bigskip

We will first present the language, its syntax and semantics, in \cref{sec:lang} and the type system for data-flow analysis in \cref{sec:TypeSys}.
Then we will present the soundness of the type system in \cref{sec:Soundness}, and lastly we will conclude in \cref{sec:Conc}.


\end{document}
