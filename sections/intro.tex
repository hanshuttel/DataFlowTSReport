\documentclass[../../master.tex]{subfiles}
\begin{document}
\section{Introduction}

\section{Notations}
This section describes some notations used in the following section.

\paragraph{Program point}
A program point is a labelling of program fragments, allowing to distinguish each expression, and subexpressions, in the program.
We define labelled program fragments as expressions and terms as unlabelled fragments.

\paragraph{Occurrence}
An occurrence is either a syntactical or semantical element, labelled with a program point.
Syntactical occurrences are elements named at some program point, expressed as $e^p$.
Semantical occurrences are elements applied at some program point, expressed as either $\loc^p$ or $x^p$ for location or variable occurrences, respectively.

\paragraph{Set notation}
Sets a defined as upper case letters.

\paragraph{Set occurrence}
When labelling a set with a program point, it indicates that the set contains occurrences, e.g. $V^p=\{x_1^{p_1},\cdots,x_n^{p_n}\}$ where $n$ is the number of elements in the set.
\end{document}
