\item[\runa{Let}] Here, $e^p=[\mbox{let}\;x\;e_1^{p'}\;e_w^{p''}]^p$, where we have:
			\begin{itemize}
				\item $\Gamma,\Pi\vdash [\mbox{let}\;x\;e_1^{p'}\;e_2^{p''}]^p:T$,
				\item $env\vdash\left\langle [\mbox{let}\;x\;e_1^{p'}\;e_2^{p''}]^p,w,sto,p_3\right\rangle\rightarrow\left\langle v,w',sto',(L,V),p\right\rangle$,
			\end{itemize}
			and that $(w,env)\models T$ and $(w,env)\models\Gamma$.

			To conclude the $[let]$ rule, we need to analyse the premises in the collection semantics and type system.
			In the collection semantics, we have the following:
			\begin{figure}[H]
			\setlength\tabcolsep{8pt}
			\begin{tabular}{l}
				\runa{Let}
				\inference[]
				{env\vdash \left\langle e_1^{p'},sto,w,p \right\rangle \rightarrow \left\langle v',sto'',w'',(L',V'),p' \right\rangle &\\
				env[x\mapsto v]\vdash \left\langle e_2^{p''},sto'',w_3,p' \right\rangle \rightarrow \left\langle v,sto',w',(L,V),p'' \right\rangle}
				{env\vdash \left\langle [\mbox{let}\;x\;e_1^{p'}\;e_2^{p''}]^{p_3},sto,w,p \right\rangle \rightarrow \left\langle v,sto',w',(L,V),p_3 \right\rangle}\\
				Where $w_3=w''[x^{p'}\mapsto(L,V)]$\\[1cm]
			\end{tabular}
			\end{figure}
			And in the type system, we have the following:
			\begin{figure}[H]
			\setlength\tabcolsep{8pt}
			\begin{tabular}{l}
				\runa{Let}
				\inference[]
				{\Gamma;\Pi\vdash e_1^{p'}:T_1 &\\
				\Gamma,x^{p'}:T_1;\Pi\vdash e_2^{p''}:T}
				{\Gamma;\Pi\vdash [\mbox{let}\; x \; e_1^{p'} \; e_2^{p''}]^{p_3}:T}\\[1cm]
			\end{tabular}
			\end{figure}

			In the first premise, if we know that $(w,env)\models T_1$, then from the induction hypothesis, we can immediately get that $\Gamma;\Pi\vdash v':T_1$, $(w'',env)\models T_1$, $(w',env)\models\Gamma$, and $(L',V')\models T_1$.
			Where we also know that an extension of $w$ holds, and $w'$ is an extension of $w$, then to conclude the first premise we need to find out if $(w,env)\models T_1$ holds.
			Here, we know that $\Gamma;\Pi\vdash v':T_1$, and we know that if $x$ occurs in $e_2^{p''}$, then $(w,env)\models T_1$ must hold by the substitution lemma.

			In the second premise, we already know that $(w'',env)\models T$ where we know that $w''$ is an extension of $w$, and thus $w=w''\mid_p$.
			Then for the second premise to hold, we also need to show for $env[x\mapsto v']$.
			If we know that $(w'',env[x\mapsto v'])\models T$, then by our induction hypothesis we can immediately get $\Gamma,x^{p'}:T_1;\Pi\vdash v:T$, $(w',env[x\mapsto v'])\models\Gamma,x^{p'}:T_1$, $(w',env[x\mapsto v'])\models T$, and $(L,V)\models T$.
			Since $env[x\mapsto v']$ is a local extension, and we know the type and dependencies of $v'$, we also know that $(w'',env[x\mapsto v])\models T$, and can be further proven by the use of the substitution lemma.
			From this, the conclusion is follows.
