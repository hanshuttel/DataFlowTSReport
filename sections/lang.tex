\documentclass[../../master.tex]{subfiles}
\begin{document}
\section{Language}\label{sec:lang}
This section will introduce a functional programming language, based on a subset of ReScript.
The language presented here is basically a $\lambda$-calculus with bindings, pattern matching and mutability.
As the purpose of the dependency analysis is to analyse each subexpression of a program and differentiate them, the language is extended to label all expressions and subexpression.
For the sake of the rest of the paper, we call the label a program point.
When labelling a syntactical element or semantical element, we call it an occurrence, such that the syntactical occurrences makes up the program, an is thus labelled expressions, while semantical occurrences are then variables and locations.

In the rest of the paper, we make a distinction between labelled and unlabelled expressions, such that a term is a labelled expression, and unlabelled expressions are just called expressions.

The rest of this section will first formally introduce the abstract syntax for the language, and then the semantics.
The semantics introduced here is an operational big-step semantics where we add the notion of program points and a mechanism for collecting dependencies.
%The main purpose of adding these two notions is to compare it with the dependency system, introduced later, and can thus be safely removed to obtain a 

\subsection{Syntax}
We will now introduce the abstract syntax for the language with changes compared to the previous work, found in \cite{DVNicky}.
We will first introduce the syntactic categories, as also done \cite{DVNicky}, and then introduce the abstract syntax.

Let the Syntactic category $c\in\cat{Con}$ denotes the category of constants, $t\in\cat{Term}$ denote the category of terms, and $e\in\cat{Exp}$ denotes the category of expressions.
The syntactic category for program points is denoted by $p\in\cat{P}$, for variables is denoted by $x,\;f\in\cat{Var}$, and locations are denoted as $\loc\in\cat{Loc}$.
\bigskip

We introduce the notion for a category labelled with program points, such that, for \cat{Exp} we have $\cat{Exp}^P=\cat{Exp}\times\cat{P}$.
We also labels occurrences in a similar way, such that, for \cat{Exp} we have $e^p\in\cat{Exp}^P$, if $e\in\cat{Exp}$ and $p\in\cat{P}$.
%We also introduce a notion for categories that represent occurrences, where they are labelled with the category for program points \cat{P}, i.e., for expressions \cat{Exp}, we then have the category for expression occurrences $e^p\in\cat{Exp}^P$.

%We can then properly define the occurrences, where we introduce 2 types of occurrences, that is, for syntactical occurrences, denoted as $e^p\in\cat{Exp}^P$, and semantical occurrences.
%The semantical occurrences is further divided into two types, for identifiers and locations and denoted as $x^p,\;f^p\in\cat{Var}^P$ and $\loc^p\in\cat{Loc}^P$ respectively.

The formation rules is then defined, as in \cref{fig:coresyntax}:

\begin{figure}[H]
	\begin{minipage}[t]{0.45\textwidth}
		\setlength\tabcolsep{4pt}
		\begin{tabular}{>{$}l<{$}>{$}r<{$}>{$}l<{$}}
			term \; t &::= &e^p \\\\

			expression \; e &::= &x \mid c \mid t_1\;t_2 \mid \lambda x.t\\
			&| &\mbox{let} \; x \; t_1 \; t_2 \\
			&| &\mbox{let rec} \; x \; t_1 \; t_2 \\
			&| &\mbox{case} \; t_1 \; \{\pi\} \\
			&| &\mbox{ref} \; t \mid t_1 := t_2 \mid \; !t\\\\

			Patterns \; \pi &::= &(s\;t)\mid(s\;t)\;\pi\\\\

			Pattern \; s &::= &n \mid b \mid x \mid \_ \\
		\end{tabular}
	\end{minipage}
	\begin{minipage}[t]{0.45\textwidth}
		\setlength\tabcolsep{4pt}
		\begin{tabular}{>{$}l<{$}>{$}r<{$}>{$}l<{$}}
			Constant\; c &::= &n \mid b\\
			&| &PLUS \\
			&| &MINUS \\
			&| &TIMES\\
			&| &TIMES \\
			&| &EQUAL \\
			&| &LESS\\
			&| &GREATER\\
			&| &NOT \\ \\
		\end{tabular}
	\end{minipage}
	\caption{Abstract syntax}
	\label{fig:coresyntax}
\end{figure}

Some notable constructs is further explained below.

\paragraph{Abstractions}
Abstractions denotes functions with a parameter $x$ and body $t$, such that $x\in\Var$.
Here, abstractions use juxtaposition of expressions, where abstractions have exactly one argument.

\paragraph{Constants}
Constants are either natural numbers $n$, boolean values $b$, or functional constants.
For each functional constant we introduce a function $apply$, that for each constant returns the result of applying a constant to its arguments.
Note that, as the abstractions only take a single argument, the same should also apply for the $apply$ function, where if a functional takes more than one argument $apply$ should thus be used twice, as an example for $PLUS$ we have:
$$apply(apply(PLUS,2),2)=2+2$$

\paragraph{Bindings}
Bindings are immutable and are done through local declaration, which can be recursive or non-recursive by the use of the $rec$ keyword.
A binding is done on placeholder variables, such that $x,\;f\in\IVar$.
Here, if $\mbox{let}\;x\;t_1\;t_2$ has an occurrence of $x$ in $t_1$ it is not treated as a recursive call to $x$ and $x$ should thus be specified elsewhere.
If we add the $rec$ keyword, then if $x$ appears in $t_1$, it should thus be treated as a recursive call to $x$.

\paragraph{Mutability}
Mutability are handled as references, that is, for the constructs $\mbox{ref}\;t$, $t_1\;:=\;t_2$, and $!t$, for declaring a new location, writing to a location, or reading from a location, respectively.
These constructs wraps the mutability to a semantical construct for locations.
%Note that, even though bindings are immutable, a variable can still be bound to a location.
%This does not make the variable mutable, as it will then be bound to this location, and cannot be bound to another value or location.

\paragraph{Pattern matching}
The pattern matching construct matches a term $t$ with the patterns in $\pi$.
The pattern matching is done by matching on each pattern, until a match is found, in the order given by the syntax.
A pattern is of the form $(s\;t)$, with at least one pattern.
We also denote $|\pi|$ as the number of pattern that occurs in $\pi$.
For the sake of distinguishing the patterns $\pi$, for use in the proof, the patterns are numbered in the order given by the syntax, such that $(s_1\;t_1),(s_2\;t_2),\cdots,(s_n\;t_n)$ where $n=|\pi|$.

\subsubsection{Free and bound variables}
The free and bound variables are defined in the usual way for $\lambda$-calculus and can be defined as:
\begin{definition}[Free variables]\label{def:fv}
	The set of free variables is given by:
	\begin{align*}
		fv(x^p)&=\{x\}\\
		fv(c^p)&=\emptyset\\
		fv([\lambda\;y.e_1^{p'}]^p)&=fv(e_1^{p''})\backslash\{y\}\\
		fv([e_1^{p'}\;e_2^{p''}]^p)&=fv(e_1^{p'})\cup fv(e_2^{p''})\\
		fv([\mbox{let}\;y\;e_1^{p'}\;e_2^{p''}]^p)&=fv(e_1^{p'})\cup fv(e_2^{p''})\backslash\{y\}\\
		fv([\mbox{let rec}\;f\;e_1^{p'}\;e_2^{p''}]^p)&=fv(e_1^{p'})\cup fv(e_2^{p''})\backslash\{f\}\\
		fv([\mbox{case}\;e^{p'}\;\pi^{p''}]^p)&=fv(e_1^{p'})\cup fv(\pi)\\
		fv([(s\;e^{p'})\;\pi])&=fv(e^{p'})\cup fv(\pi)\backslash\tau(s)\\
		fv([(s\;e^{p'})])&=fv(e^{p'})\backslash\tau(s)\\
		fv([\mbox{ref}\;e^{p'}]^p)&=fv(e^{p'})\\
		fv([!e^{p'}]^p)&=fv(e^{p'})\\
		fv([e_1^{p'}\;:=\;e_2^{p''}]^p)&=fv(e_1^{p'})\cup fv(e_2^{p''})\\
	\end{align*}
\end{definition}

\begin{definition}[Bound variables]\label{def:bv}
	The set of bound variables is given by:
	\begin{align*}
		bv(x^p)&=\emptyset\\
		bv(c^p)&=\emptyset\\
		bv([\lambda\;y.e_1^{p'}]^p)&=bv(e_1^{p'})\cup\{y\}\\
		bv([e_1^{p'}\;e_2^{p''}]^p)&=bv(e_1^{p'})\cup bv(e_2^{p''})\\
		bv([\mbox{let}\;y\;e_1^{p'}\;e_2^{p''}]^p)&=bv(e_1^{p'})\cup bv(e_2^{p''})\cup\{y\}\\
		bv([\mbox{let rec}\;f\;e_1^{p'}\;e_2^{p''}]^p)&=bv(e_1^{p'})\cup bv(e_2^{p''})\cup\{f\}\\
		bv([\mbox{case}\;e^{p'}\;\pi^{p''}]^p)&=bv(e_1^{p'})\cup bv(\pi)\\
		bv([(s\;e^{p'})\;\pi])&=bv(e^{p'})\cup bv(\pi)\cup\tau(s)\\
		bv([(s\;e^{p'})])&=bv(e^{p'})\cup\tau(s)\\
		bv([\mbox{ref}\;e^{p'}]^p)&=bv(e^{p'})\\
		bv([!e^{p'}]^p)&=bv(e^{p'})\\
		bv([e_1^{p'}\;:=\;e_2^{p''}]^p)&=bv(e_1^{p'})\cup bv(e_2^{p''})\\
	\end{align*}
\end{definition}

\subsection{Collection semantics}\label{sec:sem}
We will now introduce the big-step semantics extended with the collection of occurrence dependencies.
This collection of occurrence dependencies is done through the dependency function $w$, defined as:

$$w:\cat{Loc}^P\cup\cat{Var}^P\rightharpoonup\Pow{\cat{Loc}^P}\times\Pow{\cat{Var}^P}$$

A lookup in the dependency function are $w(x^p)=(\{\loc^{p_1},\cdots,\loc^{p_n}\},\{x_1^{p'_1},\cdots,x_y^{p'_y}\})$.
We denote the pair retrieved from the dependency function as $(L,V)$, where $L\subseteq\Pow{\cat{Loc}^P}$ is a set of location occurrences and $V\subseteq\Pow{\cat{Var}^P}$ is a set of variable occurrences.

The dependency function $w$ can be used to derive a control flow, as all bindings of an occurrence, $u^p\in\cat{Loc}^P\cup\cat{Var}^P$, is always bound to occurrences that is evaluated before $u^p$.
This control flow can thus be used to derive an ordering $\sqsubseteq_w$ of program points, defined as:

\begin{definition}[$\sqsubseteq_w$]\label{def:popular}
	Let $w:\cat{Loc}^P\cup\cat{Var}^P\rightharpoonup\Pow{\cat{Loc}^P}\times\Pow{\cat{Var}^P}$ be a dependency function.
	Then $\sqsubseteq_w\subseteq\cat{P}\times\cat{P}$, is given by:
	$$p_2\sqsubseteq p_1$$
	iff
	\begin{itemize}
		\item $u_1^{p_1}\in dom(w)\Rightarrow\exists u_2^{p_2}\;\dot{\in}\; w(u_1^{p_1})$, or
		\item $p_1=p_2$
	\end{itemize}
	Where $u^{p}\;\dot{\in}\;w(v^{p'})$, for $u^p,v^{p'}\in\cat{Loc}^P\cup\cat{Var}^P$, is given by:
	if $w(v^{p'})=(L,V)$ then $u^{p}\in L\vee u^{p}\in V$
\end{definition}

This definition is used to define an order between two program points, in respect to the dependency function $w$.

\begin{definition}[Partial order of $w$]
	We say that $w$ is po-pular iff $\sqsubseteq_w$ is a partial order.
\end{definition}
%We then say that $w$ is a partial order, called po-polar, if the order relation $\sqsubseteq_w$ gives a partial order relation.

With the dependency function defined, the transition system can now be presented:
%we can now present the transition system, which is similarly defined in (previous paper), and are of the form:
\begin{align*}
env\vdash\left\langle e^{p'},sto,w,p\right\rangle\rightarrow\left\langle v,sto',w',(L,V),p''\right\rangle
\end{align*}
Where $env\in\cat{Env}$ is the environment and $sto\in\cat{Sto}$ is the store.
This should be read as, given the store $sto$, dependency function $w$, and the previous program point $p$, the occurrence $e^{p'}$ evaluates to a value $v$, an updated store $sto'$, an updated dependency function $w'$, the dependency pair $(L,V)$, and the program point $p''$ reached after evaluating $e^{p'}$, given the bindings in the environment $env$.

The environment keeps the bindings currently known, such that:
$$\cat{Env}=\cat{Var}\rightharpoonup\cat{Values}$$
We also define a function that, for a given value, return the set of variables bound to this variable.
\begin{definition}[inverse env]
	Let $v$ be a value, $env$ be an environment, the inverse function $env{-1}$ is thus:
	$$env^{-1}(v)=\{x\in dom(env)\mid env(x)=v\}$$
\end{definition}


The store keeps the location bindings known, and a placeholder $next$ that represent the next free location.

$$\cat{Sto}=\cat{Loc}\cup\{next\}\rightharpoonup\cat{Values}$$

Where we consider the following as values:
\begin{itemize}
	\item All constants are values.
	\item Locations are values.
	\item Closures, \InfVal{x,e,env} are values.
	\item Recursive closures, \InfVal{f,x,e,env}, are values.
	\item Unit values, $()$, are values.
\end{itemize}
The set of values is denoted as \cat{Values} and defined as:
\begin{definition}[Values]
	A value $v\in\cat{Values}$ is an expression given by the following formation rule:
	$$v::=c\mid\loc\mid\left\langle x^{p},e^{p'},env\right\rangle\mid\left\langle x^{p},f^{p'},e^{p''},env\right\rangle\mid ()$$
\end{definition}

We also assume the existence of a function $new:\cat{Loc}\rightharpoonup\cat{Loc}$, which takes a location and finds the next location.
This function is only used on the placeholder location, $next$, to find the next free location, i.e. a location that is not already bound in $sto$.

Note that, in the semantics we refer to syntactic occurrences, $e^p$, directly instead of terms.
This is done as many transition rules, when evaluating, evaluates to the program point of the occurrence.
\bigskip

Before introducing some transition rules, we first need to introduce a function that, for a given occurrence $x^p$ or $\loc^p$, finds the closest occurrence.
It should be noted that, for an occurrence $x^p$, the closest occurrence should be for the same variable, and the same holds for locations.

\begin{definition}[$inf_p$ function]
	Let $x^p$ be a occurrence and $w$ be a dependency function that is po-pular, then $\inf(x^p,w)=x^{p'}$ is as follows:
	\begin{align*}
		\inf(x^p,w)=x^q \;\;\; \mbox{if} \; q=\inf\{q'\mid x^{q'}\in dom(w), q'\sqsubseteq_w p\}
	\end{align*}
\end{definition}
\bigskip

\todo[inline]{Rewrite rest of this section}
Only a couple of interesting transition rules will be shown here, the rest can be found in \cref{App:ColSem}.
The first transition rule, $\runa{Con}$, is the simplest rule that shows part of the transition system.
Here, as a constant is directly evaluated there is no side effects, the pair $(L,V)$ is empty, and the program point reached after evaluating $c^p$ is $p$.

\todo[inline]{$\runa{Var}$ description needs a rewrite}
In the second rule, $\runa{Var}$, the environment and dependency function comes into play, as the value and dependencies of the variable needs to be looked up.
The value for the variable is easy, as it is directly looked up in the environment.
To get the dependencies of a variable, the closest occurrence needs to be found first.
This is done through the $inf_p$ function, which finds the closest occurrence for the same variable in $w$, and returns the program point.
With the given program point, the dependencies for this variable, at the given program point, can be looked up in $w$.
Since the variable occurrence, $x^{p'}$, is used, it is added to the set of variable occurrences.

The third rule, $\runa{let}$, is for when a local declaration is reached, here we add the information from the local declaration in both the environment and dependency function.
The dependency function is updated with the liveness pair from evaluating the local declaration for the variable occurrence $x^{p'}$, where the program point is the point reached after evaluating the local declaration.

The next two rules shown, $\runa{Loc new}$ and $\runa{Loc write}$, are for handling creating a new reference and writing to a reference.
The $\runa{Loc new}$ rule evaluated the body to a value $v$ and returns a location.
This location, $\loc$, is retrieved from the placeholder location $next$ in $sto$ and $\loc$ is bound to $v$ in $sto$.
The dependency function is also updated with the liveness information from evaluating the body, and is bound to the occurrence for $\loc$ at the program point $p''$.

For $\runa{Loc write}$, first the two terms are evaluated, where first term needs to be a location.
The location $\loc$ is bound to the value from evaluating the second term, in $sto$ and the $w$ is updated with the liveness pair from the second term, to the occurrence for $\loc$ at point $p_3$.
Since the value of a write expression, in ReScript, does not return a value, the value returned here is the unit value $()$.


\begin{figure}[H]
	\setlength\tabcolsep{8pt}
	\begin{tabular}{l}
		\InfName{Const}\\[0.2cm]
			\inference[]{}
				{env\vdash \left\langle c^{p'},sto,w,p \right\rangle \rightarrow \left\langle c,sto,w,(\emptyset,\emptyset),p' \right\rangle}\\[1cm]
			
		\InfName{Var}\\[0.2cm]
			\inference[]{}
				{env\vdash \left\langle x^{p'},sto,w,p \right\rangle \rightarrow \left\langle v,sto,w,(L,V\cup\{x^{p'}\}),p' \right\rangle}\\
				Where $env(x)=v$, $p''=\inf_{p} (x^{p'},w)$, and $w(x^{p''})=(L,V)$\\[1cm]

		\InfName{Let}\\[0.2cm]
			\inference[]
			{env\vdash \left\langle e_1^{p'},sto,w,p \right\rangle \rightarrow \left\langle v,sto',w',(L,V),p' \right\rangle &\\
			env[x\mapsto v]\vdash \left\langle e_2^{p''},sto',w'',p' \right\rangle \rightarrow \left\langle v',sto'',w_3,(L',V'),p'' \right\rangle}
			{env\vdash \left\langle [\mbox{let}\;x\;e_1^{p'}\;e_2^{p''}]^{p_3},sto,w,p \right\rangle \rightarrow \left\langle v',sto'',w_3,(L',V'),p_3 \right\rangle}\\
		Where $w''=w'[x^{p'}\mapsto(L,V)]$\\[1cm]

		\InfName{Loc\;new}\\[0.2cm]
			\inference[]
				{env \vdash \left\langle e^{p'},sto,w,p \right\rangle \rightarrow \left\langle v,sto',w',(L,V),p' \right\rangle}
				{env\vdash \left\langle [\mbox{ref}\;e^{p'}]^{p''},sto,w,p \right\rangle \rightarrow \left\langle \loc,sto''[\loc\mapsto v],w'',(L,V),p'' \right\rangle}\\
			Where $\loc = next$, $sto''=sto'[next\mapsto new\;\loc]$ and $w''=w'[\loc^{p''}\mapsto (L,V)]$\\[1cm]

		\InfName{Loc\;read}\\[0.2cm]
			\inference[]
				{env \vdash \left\langle e^{p'},sto,w,p \right\rangle \rightarrow \left\langle \loc,sto',w',L,V,p' \right\rangle}
				{env\vdash \left\langle [!e^{p'}]^{p''},sto,w,p \right\rangle \rightarrow \left\langle v,sto',w',L\cup\{\loc^{p''}\},V,p'' \right\rangle}\\
			Where $sto'(\loc)=v$\\[1cm]


		\InfName{Loc\;write}\\[0.2cm]
			\inference[]
				{env \vdash \left\langle e_1^{p'},sto,w,p \right\rangle \rightarrow \left\langle \loc,sto',w',(L,V),p' \right\rangle &\\
				env \vdash \left\langle e_2^{p''},sto',w',p' \right\rangle \rightarrow \left\langle v,sto'',w'',(L',V'),p'' \right\rangle}
				{env\vdash \left\langle [e_1^{p'}:=e_2^{p''}]^{p_3},sto,w,p \right\rangle \rightarrow \left\langle (),sto_3,w_3,(L,V),p_3 \right\rangle}\\
				Where $sto_3=sto''[\loc\mapsto v]$ and $w_3=w''[\loc^{p_3}\mapsto(L',V')]$\\[1cm]
	\end{tabular}
	\caption{Collection semantics}
	\label{fig:ColSem}
\end{figure}
\end{document}
