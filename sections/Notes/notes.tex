\documentclass[../../master.tex]{subfiles}
\begin{document}
\section{Notes}

\subsection{Free and bound variables}
We denote $\tau(s)$, for a pattern $s$, as
$$
	\tau(s)=
		\left\{\begin{matrix}
			\{x\} & \mbox{if}\;s=x\\ 
			\emptyset & \mbox{otherwise}
		\end{matrix}\right.
$$

\begin{definition}[Free variables]\label{def:fv}
	The set of free variables is given by:
	\begin{align*}
		fv(x^p)&=\{x\}\\
		fv(c^p)&=\emptyset\\
		fv([\lambda\;y.e_1^{p'}]^p)&=fv(e_1^{p''})\backslash\{y\}\\
		fv([e_1^{p'}\;e_2^{p''}]^p)&=fv(e_1^{p'})\cup fv(e_2^{p''})\\
		fv([\mbox{let}\;y\;e_1^{p'}\;e_2^{p''}]^p)&=fv(e_1^{p'})\cup fv(e_2^{p''})\backslash\{y\}\\
		fv([\mbox{let rec}\;f\;e_1^{p'}\;e_2^{p''}]^p)&=fv(e_1^{p'})\cup fv(e_2^{p''})\backslash\{f\}\\
		fv([\mbox{case}\;e^{p'}\;\pi^{p''}]^p)&=fv(e_1^{p'})\cup fv(\pi)\\
		fv([(s\;e^{p'})\;\pi])&=fv(e^{p'})\cup fv(\pi)\backslash\tau(s)\\
		fv([(s\;e^{p'})])&=fv(e^{p'})\backslash\tau(s)\\
		fv([\mbox{ref}\;e^{p'}]^p)&=fv(e^{p'})\\
		fv([!e^{p'}]^p)&=fv(e^{p'})\\
		fv([e_1^{p'}\;:=\;e_2^{p''}]^p)&=fv(e_1^{p'})\cup fv(e_2^{p''})\\
	\end{align*}
\end{definition}

\begin{definition}[Bound variables]
	The set of bound variables is given by:
	\begin{align*}
		bv(x^p)&=\emptyset\\
		bv(c^p)&=\emptyset\\
		bv([\lambda\;y.e_1^{p'}]^p)&=bv(e_1^{p'})\cup\{y\}\\
		bv([e_1^{p'}\;e_2^{p''}]^p)&=bv(e_1^{p'})\cup bv(e_2^{p''})\\
		bv([\mbox{let}\;y\;e_1^{p'}\;e_2^{p''}]^p)&=bv(e_1^{p'})\cup bv(e_2^{p''})\cup\{y\}\\
		bv([\mbox{let rec}\;f\;e_1^{p'}\;e_2^{p''}]^p)&=bv(e_1^{p'})\cup bv(e_2^{p''})\cup\{f\}\\
		bv([\mbox{case}\;e^{p'}\;\pi^{p''}]^p)&=bv(e_1^{p'})\cup bv(\pi)\\
		bv([(s\;e^{p'})\;\pi])&=bv(e^{p'})\cup bv(\pi)\cup\tau(s)\\
		bv([(s\;e^{p'})])&=bv(e^{p'})\cup\tau(s)\\
		bv([\mbox{ref}\;e^{p'}]^p)&=bv(e^{p'})\\
		bv([!e^{p'}]^p)&=bv(e^{p'})\\
		bv([e_1^{p'}\;:=\;e_2^{p''}]^p)&=bv(e_1^{p'})\cup bv(e_2^{p''})\\
	\end{align*}
\end{definition}


\subsection{Basis}
This section introduces the basis for the type checker, as such the type bases er assumptions.
\begin{definition}[Type Base for aliasing]
	The type base $\kappa^0$ is a partition of $\cat{Var}\cup\cat{IVar}$, such that:
	$$\cat{Var}\subseteq\bigcup_{i\in I}(\kappa_i^0)$$
	and that $\kappa_i^0\neq\kappa_j^0$ for all $i\neq j$
\end{definition}
The type base for aliasing is an assumption for which variables that shares the same internal variable, and as such, shares the same location.
As such, if we have a an assumption for a variable $x$, such that $\{x\}=\kappa_i$, then $x$ would not be an alias to a location.
On the otherhand if we have $\{y,\nu y\}=\kappa_j$, for a variable $y$ and internal variable $\nu y$, then $y$ is an alias for $\nu y$.

\subsection{Type environment}
\begin{definition}[Type Environment]
	A type environment $\Gamma$ is a partial function $\Gamma:Id^P\rightharpoonup TYPES$
\end{definition}
As such, $\Gamma$ represents the dependencies for variables and internal variables.

\begin{definition}[Approximated order of program points]
	An approximated order of program points $\Pi$, such that: 
	\begin{itemize}
		\item  $p\sqsubseteq p'\in\Pi$
		\item The order of program points is transitive, such that if $p\sqsubseteq p'\in\Pi$ and $p'\sqsubseteq p''\in\Pi$ then $p\sqsubseteq p''\in\Pi$.
	\end{itemize}
\end{definition}
The intuition behind an approximated order of program points is the execution order, and should as such follow all possible evaluation paths for a given program.

\subsection{Agreement}
The first agreement we present here is the agreement for the type environment $\Gamma$ and dependency function $w$.
The $\Gamma$ contains both local dependencies for variable occurrences and global dependencies for internal variable occurrences, while $w$ contains global dependencies for variable and location occurrences.

As such. it is clear that $\Gamma$ is not a direct model of $w$, but models the current state that is evaluated, thus it models both the environment, store, and dependency function.
This is clear, as both the environment and type environment both only contains local information for variables.

On the other hand, locations are treated as global information in the type system, which are represented as internal variables, as such if a location exists in the store it would thus the equivalent would exists in the type environment.
Since the type environment contains only occurrences, locations and internal variables can thus be compared by program points, as locations and internal variables are created and updated at the same program points.
This relation is done by comparing between between the store, dependency function and type environment.

Lastly, since the dependency function and type environment collects information about dependencies, the dependencies ar then compared.
Since there a both local and global information, and the dependency function collects all information globally, the dependency comparison is done only on the the information  for occurrences that exists in the domain of both $w$ and $\Gamma$.

\begin{definition}[Environment agreement]\label{def:EnvAgree}
	Let $w$ be a dependency function, $env$ be an environment, $sto$ be the a store, and $\Gamma$ be a type environment.
	We say that:
	$$(w,env,sto)\models\Gamma$$
	if 
	\begin{itemize}
		\item $\forall x\in dom(env)\Rightarrow\exists x^p\in dom(w)$
		\item $\forall x\in dom(env).x^p\in dom(w)\Rightarrow \exists x^p\in dom(\Gamma)$
		\item $\forall x^p\in dom(w).x^p\in dom(\Gamma)\Rightarrow w(x^p)=(L,V)\wedge\Gamma(x^p)=T.(w,env,(L,V))\models T$
		\item $\forall \loc\in dom(sto)\Rightarrow\exists \loc^p\in dom(w)$
		\item $\forall \loc\in dom(sto)\Rightarrow\exists \nu x.\forall p\in\{p'\mid\loc^{p'}\in dom(w)\}$ then $\nu x^p\in dom(\Gamma)$
		\item $\forall \loc^p \in dom(w).\exists\nu x^{p}\in dom(\Gamma)\Rightarrow w(\loc^p)=(L,V)\wedge\Gamma(\nu x^{p})=T.(w,env,(L,V))\models T$
	\end{itemize}
\end{definition}
In \cref{def:EnvAgree}, the agreement contains a lot of conditions, but in essence it states:
\begin{description}
	\item[variables] If a variable exists in $env$, then there exists an occurrence of that variable in both $w$ and $\Gamma$.
		And for all variable occurrences, $x^p$, that both $w$ and $\Gamma$ knows about, the dependency and type for $x^p$ agrees.
	\item[locations] The agreement is similar for locations, but there is an extra step as the type system represents locations as internal variables.
		The comparison is thus done by comparing program points and since this is global, the comparison is done for all program points that exists for a a location $\loc$ in $w$.
\end{description}

\begin{definition}[Type agreement]
	Let $w$ be a dependency function, $env$ be an environment, $(L,V)$ be a dependency pair, and $T$ be a type.
	We say that:
	$$(w,env,(L,V))\models T$$
	if
	\begin{itemize}
		\item $T=T_1\rightarrow T_2$ then:
		\begin{itemize}
			\item $(w,env,(L,V))\models T_2$
		\end{itemize}
		\item $T=(\delta,\kappa)$ then:
		\begin{itemize}
			\item $(env,(L,V))\models\delta$
			\item $(w,env)\models\kappa$
		\end{itemize}
	\end{itemize}
\end{definition}

\begin{definition}[Dependency agreement]
	Let $env$ be an environment, $(L,V)$ be a dependency pair, and $\delta$ be a set of variables.
	We say that:
	$$(env,(L,V))\models\delta$$
	if
	\begin{itemize}
		\item $V\subseteq\delta$,
		\item For all $\loc^p\in L$ where $\exists x\in dom(env).env(x)=\loc$, we then have $\{x\in dom(env)\mid env(x)=\loc\}\subseteq \kappa_i^0$ for a $\kappa_i^0\in\delta$
		\item For all $\loc^p\in L$ where $\not\exists x \in dom(env).env(x)=\loc$ then there exists a $\kappa_i^0\in\delta$ such that $\kappa_i^0\subseteq\cat{IVar}$
	\end{itemize}
\end{definition}

\begin{definition}[Alias agreement]
	Let $env$ be an environment, and $\kappa$ be an alias set.
	We say that:
	$$(env,(L,V))\models\kappa$$
	if
	\begin{itemize}
		\item For all $\loc^p\in dom(w)$ where $\exists x\in dom(env).env(x)=\loc$, we then have $\{x\in dom(env)\mid env(x)=\loc\}\subseteq \kappa_i^0$ for a $\kappa_i^0\in\kappa$
		\item For all $\loc^p\in dom(w)$ where $\not\exists x \in dom(env).env(x)\neq\loc$ then there exists a $\kappa_i^0\in\kappa$ such that $\kappa_i^0\subseteq\cat{IVar}$
	\end{itemize}
\end{definition}

\subsection{Judgement}
\begin{definition}[Environment judgement]
	Let $env$ be an environment, $\Gamma$ be a type environment, and $\Pi$ be the approximated order of program points.
	We say that:
	$$\Gamma;\Pi\vdash env$$
	if 
	\begin{itemize}
		\item For all $x\in dom(env)$ and for all $x^p\in dom(\Gamma).\Gamma(x^p)=T_x$ then 
			$$\Gamma,\Pi\vdash env(x):T_x$$
	\end{itemize}
\end{definition}

\subsection{Type rules for values}
For the sake of proving the type system, we present type rules for values.
The formation rules for values can be given by:
$$v::=c\mid\loc\mid\left\langle x^{p},e^{p'},env\right\rangle\mid\left\langle x^{p},f^{p'},e^{p''},env\right\rangle\mid ()$$
Where the type rules is given in \cref{fig:ValTypeRules}.
\begin{figure}[H]
	\setlength\tabcolsep{8pt}
	\begin{tabular}{l}
		\hline\\
		\runa{Constant}\\[0.4cm]
			\inference[]{}
				{\Gamma;\Pi\vdash  c:(\emptyset, \emptyset)}\\[1cm]

		\runa{Location}\\[0.4cm]
			\inference[]{}
				{\Gamma;\Pi\vdash  \loc:(\delta, \kappa)}\\[1cm]

		\runa{Closure}\\[0.4cm]
			\inference[]
				{
					\Gamma;\Pi\vdash env \\
					\Gamma,x^{p}:T_1;\Pi\vdash e^{p'}:T_2
				}
				{\Gamma;\Pi\vdash \left\langle x^{p}, e^{p'}, env \right\rangle^{p''}:T_1\rightarrow T_2}\\[1cm]

		\runa{Recursive closure}\\[0.4cm]
			\inference[]
				{
					\Gamma;\Pi\vdash env \\
					\Gamma,x^{p}:T_1,f^{p'}:T_1\rightarrow T_2;\Pi\vdash e^{p''}:T_2
				}
				{\Gamma;\Pi\vdash \left\langle x^{p}, f^{p'}, e^{p''}, env \right\rangle^{p_3}:T_1\rightarrow T_2}\\[1cm]

		\runa{Unit}\\[0.4cm]
			\inference[]{}
			{\Gamma;\Pi\vdash  ():T}\\[0.5cm]
		\hline
	\end{tabular}
	\caption{Type rules for values}
	\label{fig:ValTypeRules}
\end{figure}

\subsection{Lemma}
\begin{lemma}[History]\label{lemma:His}
	If 
	$$env\vdash\left\langle e^{p'},w,sto,p\right\rangle\rightarrow\left\langle v,w',sto',(L,V),p''\right\rangle$$
		and $x^{p_1}\in dom(w')\backslash dom(w)$ then $x^{p_1}\notin fv(e)$
\end{lemma}
\begin{proof}
	The proof proceeds by induction on the height of the derivation tree for $env\vdash\left\langle e^{p'},w,sto,p\right\rangle\rightarrow\left\langle v,w',sto',(L,V),p''\right\rangle$.
	\begin{description}
		\item[\runa{Cons}] Here $e^{p'}=c^{p'}$, where
		$$env\vdash\left\langle c^{p'},w,sto,p\right\rangle\rightarrow\left\langle c,w',sto',(\emptyset,\emptyset),p'\right\rangle$$
		Where $w=w'$ and $sto'=sto$.
		This case follows immediate as $dom(w')\backslash dom(w)=\emptyset$.

		\item[\runa{Var}] Here $e^{p'}=x^{p'}$, where
		$$env\vdash\left\langle x^{p'},w,sto,p'\right\rangle\rightarrow\left\langle v,w',sto',(L,V),p'\right\rangle$$
		Where $w=w'$, $sto'=sto$, $env(x)=v$, $inf_{p'}(x,w)=p''$, and $w(x^{p''})=(L,V)$.
		This case follows immediate as $dom(w')\backslash dom(w)=\emptyset$.
	\end{description}
	Next, follows the induction step:

	\begin{description}
		\item[\runa{Abs}] Here $e^{p'}=[\lambda y.e_1^{p''}]^{p'}$, where
		$$env\vdash\left\langle [\lambda y.e_1^{p''}]^{p'},w,sto,p\right\rangle\rightarrow\left\langle v,w',sto',(\emptyset,\emptyset),p'\right\rangle$$
		Where $w=w'$, $sto'=sto$, and $v=\left\langle y,e_1^{p''},env\right\rangle$.
		This case follows immediate as $dom(w')\backslash dom(w)=\emptyset$.
		
		\item[\runa{App}] Here $e^{p'}=[e_1^{p_1}\;e_2^{p_2}]^{p'}$, where
		\begin{figure}[H]
			\setlength\tabcolsep{8pt}
			\begin{tabular}{l}
					\inference[]
					{env \vdash \left\langle e_1^{p'},sto,w,p \right\rangle \rightarrow \left\langle v,sto',w',(L,V),p' \right\rangle &\\
					env \vdash \left\langle e_2^{p''},sto',w',p' \right\rangle \rightarrow \left\langle v',sto'',w'',(L',V'),p'' \right\rangle &\\
					env[y\mapsto v'] \vdash \left\langle e_3^{p_1},sto'',w_3,p'' \right\rangle \rightarrow \left\langle v'',sto_3,w_4,(L'',V''),p_1 \right\rangle}
					{env\vdash \left\langle [e_1^{p'}\;e_2^{p''}]^{p_3},sto,w,p \right\rangle \rightarrow \left\langle v'',sto_3,w_4,(L\cup L'',V\cup V''),p_1 \right\rangle}\\
			\end{tabular}
		\end{figure}
		Where $v=\left\langle y,e_3^{p_1},env\right\rangle$ and $w_3=w''[y^{p''}\mapsto(L',V')]$.
		By our induction hypothesis, we have:
		\begin{itemize}
			\item if $x^{p_3}\in dom(w')\backslash dom(w)$ then $x\notin fv(e_1^{p'})$
			\item if $x^{p_3}\in dom(w'')\backslash dom(w')$ then $x\notin fv(e_2^{p''})$
			\item if $x^{p_3}\in dom(w_4)\backslash dom(w_3)$ then $x\notin fv(e_3^{p_1})$
		\end{itemize}
		To show that this holds, we need to show for the extension \runa{App} introduces to $w$.
		By \cref{def:fv}, we know that the first premise evaluates to a closure, and $fv([e_1^{p_1}\;e_2^{p_2}]^{p'})=fv(e_1^{p_1})\cup fv(e_2^{p_2})$.
		Since we know that $y\notin fv(e_1^{p_1})$, and all variables are destinct, we also know that $y\notin fv([e_1^{p_1}\;e_2^{p_2}]^{p'})$.
		We can thus conclude that:
		\begin{itemize}
			\item if $x^{p_3}\in dom(w_4)\backslash dom(w)$ then $x\notin fv(e^{p'})$
		\end{itemize}

	\item[\runa{Let}] Here $e^{p'}=[\mbox{let}\;x\;e_1^{p_1}\;e_2^{p_2}]^{p'}$, where
		\begin{figure}[H]
			\setlength\tabcolsep{8pt}
			\begin{tabular}{l}
			\inference[]
				{env\vdash \left\langle e_1^{p'},sto,w,p \right\rangle \rightarrow \left\langle v,sto',w',(L,V),p' \right\rangle &\\
				env[y\mapsto v]\vdash \left\langle e_2^{p''},sto',w'',p' \right\rangle \rightarrow \left\langle v',sto'',w_3,(L',V'),p'' \right\rangle}
				{env\vdash \left\langle [\mbox{let}\;y\;e_1^{p'}\;e_2^{p''}]^{p_3},sto,w,p \right\rangle \rightarrow \left\langle v',sto'',w_3,(L',V'),p_3 \right\rangle}
			\end{tabular}
		\end{figure}
		Where $w''=w'[y^{p'}\mapsto(L,V)]$.
		By our induction hypothesis, we have:
		\begin{itemize}
			\item if $x^{p_3}\in dom(w')\backslash dom(w)$ then $x\notin fv(e_1^{p'})$
			\item if $x^{p_3}\in dom(w_3)\backslash dom(w'')$ then $x\notin fv(e_2^{p''})$
		\end{itemize}
		Then by \cref{def:fv}, we must also have
		\begin{itemize}
			\item if $x^{p_3}\in dom(w_3)\backslash dom(w)$ then $x\notin fv(e^{p'})$
		\end{itemize}
		This also hold true for the extension, $w'[y^{p'}\mapsto(L,V)]$, since by \cref{def:fv} we know that $y\notin fv(e^{p'})$.

	\item[\runa{Let rec}] Here $e^{p'}=[\mbox{let rec}\;x\;e_1^{p_1}\;e_2^{p_2}]^{p'}$, where
		\begin{figure}[H]
			\setlength\tabcolsep{8pt}
			\begin{tabular}{l}
			\inference[]
				{env\vdash \left\langle e_1^{p'},sto,w,p \right\rangle \rightarrow \left\langle v,sto',w',(L,V),p' \right\rangle &\\
				env[f\mapsto\left\langle x,f,e_1^{p'},env''\right\rangle]\vdash \left\langle e_2^{p''},sto',w',p' \right\rangle \rightarrow \left\langle v',sto'',w'',(L',V'),p'' \right\rangle}
				{env\vdash \left\langle [\mbox{let rec}\;f\;e_1^{p'}\;e_2^{p''}]^{p_3},sto,w,p \right\rangle \rightarrow \left\langle v',sto,w_3,(L',V'),p_3 \right\rangle}
			\end{tabular}
		\end{figure}
		Where $v=\left\langle y,e_1^{p'},env''\right\rangle$ and $w_3=w''[f^{p''}\mapsto(L',V')]$.
		By our induction hypothesis, we have:
		\begin{itemize}
			\item if $x^{p_3}\in dom(w')\backslash dom(w)$ then $x\notin fv(e_1^{p'})$
			\item if $x^{p_3}\in dom(w_3)\backslash dom(w'')$ then $x\notin fv(e_2^{p''})$
		\end{itemize}
		Then by \cref{def:fv}, we must also have
		\begin{itemize}
			\item if $x^{p_3}\in dom(w_3)\backslash dom(w)$ then $x\notin fv(e^{p'})$
		\end{itemize}
		This also hold true for the extension, $w'[f^{p'}\mapsto(L,V)]$, since by \cref{def:fv} we know that $f\notin fv(e^{p'})$.

	\item[\runa{Case}] Here $e^{p'}=[\mbox{case}\;e_1^{p_1}\;\pi^{p_2}]^{p'}$, where
		\begin{figure}[H]
			\setlength\tabcolsep{8pt}
			\begin{tabular}{l}
			\inference[]
				{env \vdash \left\langle e_1^{p_1},sto,w,p \right\rangle \rightarrow \left\langle v,sto',w',(L,V),p_1 \right\rangle &\\
				env \vdash \left\langle (v,(L,V),\pi^{p_2}),sto',w',p_1 \right\rangle \rightarrow \left\langle v',sto'',w'',(L',V'),p_2 \right\rangle}
				{env\vdash \left\langle [\mbox{case}\;e_1^{p_1}\;\pi^{p_2}]^{p'},sto,w,p \right\rangle \rightarrow \left\langle v',sto'',w'',(L\cup L',V\cup V'),p' \right\rangle}
			\end{tabular}
		\end{figure}
		By our induction hypothesis, we have:
		\begin{itemize}
			\item if $x^{p_3}\in dom(w')\backslash dom(w)$ then $x\notin fv(e_1^{p'})$
			\item if $x^{p_3}\in dom(w'')\backslash dom(w')$ then $x\notin fv(\pi^{p''})$
		\end{itemize}
		Then by \cref{def:fv}, we must also have
		\begin{itemize}
			\item if $x^{p_3}\in dom(w'')\backslash dom(w)$ then $x\notin fv(e^{p'})$
		\end{itemize}

	\item[\runa{Match 1}] Here $e^{p'}=[(v,(L,V),(s\;e_1^{p_1})\;\pi^{p_2}]^{p'}$, where
		\begin{figure}[H]
			\setlength\tabcolsep{8pt}
			\begin{tabular}{l}
			\inference[]
				{env\sigma \vdash \left\langle e_1^{p_1},sto,w',p \right\rangle \rightarrow \left\langle v,sto',w'',(L,V),p_1 \right\rangle}
				{env\vdash \left\langle [(v,(L,V),(s\;e_1^{p_1})\;\pi^{p_2})]^{p'},sto,w',p \right\rangle \rightarrow \left\langle v,sto',w'',(L,V),p' \right\rangle}
			\end{tabular}
		\end{figure}
		Where $match(v,s)=\sigma$, $\phi=match_w(s,(L,V))$ and $w'=w\phi$.
		By our induction hypothesis, we have:
		\begin{itemize}
			\item if $x^{p_3}\in dom(w'')\backslash dom(w')$ then $x\notin fv(e_1^{p_1})$
		\end{itemize}
		Then by \cref{def:fv}, for $fv(e^{p'})$, we know that it holds for all $x\notin\tau(s)$.
		From \cref{def:fv} we know that for all $x\in\tau(s)$, then $fv([(v,(L,V),(s\;e^{p_1})\;\pi^{p_2})]^{p'})= fv(e^{p_1}) \cup fv(\pi^{p_2})\backslash\tau(s)$, as such the we can conclude that:
		\begin{itemize}
			\item if $x^{p_3}\in dom(w'')\backslash dom(w)$ then $x\notin fv(e^{p'})$
		\end{itemize}

	\item[\runa{Match 2}] Here $e^{p'}=[(v,(L,V),(s\;e_1^{p_1}))]^{p'}$, where
		\begin{figure}[H]
			\setlength\tabcolsep{8pt}
			\begin{tabular}{l}
			\inference[]
				{env\sigma \vdash \left\langle e_1^{p_1},sto,w',p \right\rangle \rightarrow \left\langle v,sto',w'',(L,V),p' \right\rangle}
				{env\vdash \left\langle [(v,(L,V),(s\;e_1^{p_1}))]^{p_3},sto,w',p \right\rangle \rightarrow \left\langle v,sto',w'',(L,V),p_3 \right\rangle}
			\end{tabular}
		\end{figure}
		Where $match(v,s)=\sigma$, $\phi=match_w(s,(L,V))$ and $w'=w\phi$.
		By our induction hypothesis, we have:
		\begin{itemize}
			\item if $x^{p_3}\in dom(w'')\backslash dom(w')$ then $x\notin fv(e_1^{p_1})$
		\end{itemize}
		Then by \cref{def:fv}, for $fv(e^{p'})$, we know that it holds for all $x\notin\tau(s)$.
		From \cref{def:fv} we know that for all $x\in\tau(s)$, then $fv([(v,(L,V),(s\;e^{p_1})\;\pi^{p_2})]^{p'})= fv(e^{p_1}) \cup fv(\pi^{p_2})\backslash\tau(s)$, as such the we can conclude that:
		\begin{itemize}
			\item if $x^{p_3}\in dom(w'')\backslash dom(w)$ then $x\notin fv(e^{p'})$
		\end{itemize}

	\item[\runa{Match $\perp$}] Here $e^{p'}=[(v,(L,V),(s\;e_1^{p_1})\;\pi^{p_2}]^{p'}$, where
		\begin{figure}[H]
			\setlength\tabcolsep{8pt}
			\begin{tabular}{l}
			\inference[]
				{env \vdash \left\langle (v,(L,V),\pi^{p''}),sto,w,p \right\rangle \rightarrow \left\langle v,sto',w',(L,V),p'' \right\rangle}
				{env\vdash \left\langle [(v,(L,V),(s\;e^{p'})\pi^{p''})]^{p_3},sto,w,p \right\rangle \rightarrow \left\langle v,sto',w',(L,V),p_3 \right\rangle}
			\end{tabular}
		\end{figure}
		Where $match(v,s)=\perp$
		By our induction hypothesis, we have:
		\begin{itemize}
			\item if $x^{p_3}\in dom(w'')\backslash dom(w')$ then $x\notin fv(\pi^{p_2})$
		\end{itemize}
		From \cref{def:fv} we know that is must also hold for:
		\begin{itemize}
			\item if $x^{p_3}\in dom(w'')\backslash dom(w)$ then $x\notin fv(e^{p'})$
		\end{itemize}

	\item[\runa{Ref}] Here $e^{p'}=[\mbox{ref}\;e_1^{p_1}]^{p'}$, where
		\begin{figure}[H]
			\setlength\tabcolsep{8pt}
			\begin{tabular}{l}
			\inference[]
				{env \vdash \left\langle e^{p_1},sto,w,p \right\rangle \rightarrow \left\langle v,sto',w',(L,V),p_1 \right\rangle}
				{env\vdash \left\langle [\mbox{ref}\;e^{p_1}]^{p'},sto,w,p \right\rangle \rightarrow \left\langle \loc,sto''[\loc\mapsto v],w'',(\emptyset,\emptyset),p' \right\rangle}
			\end{tabular}
		\end{figure}
		Where $sto''=sto'[next\mapsto new\;\loc]$ and $w''=w'[\loc^{p'}\mapsto (L,V)]$.
		By our induction hypothesis, we have:
		\begin{itemize}
			\item if $x^{p_3}\in dom(w')\backslash dom(w)$ then $x\notin fv(e_1^{p_1})$
		\end{itemize}
		Then by \cref{def:fv}, we must also have
		\begin{itemize}
			\item if $x^{p_3}\in dom(w')\backslash dom(w)$ then $x\notin fv(e^{p'})$
		\end{itemize}

	\item[\runa{Ref read}] Here $e^{p'}=[!e_1^{p_1}]^{p'}$, where
		\begin{figure}[H]
			\setlength\tabcolsep{8pt}
			\begin{tabular}{l}
			\inference[]
				{env \vdash \left\langle e^{p_1},sto,w,p \right\rangle \rightarrow \left\langle \loc,sto',w',(L,V),p_1 \right\rangle}
				{env\vdash \left\langle [!e^{p_1}]^{p'},sto,w,p \right\rangle \rightarrow \left\langle v,sto',w',(L\cup\{\loc^{p'}\},V),p' \right\rangle}
			\end{tabular}
		\end{figure}
		Where $sto'(\loc)=v$.
		By our induction hypothesis, we have:
		\begin{itemize}
			\item if $x^{p_3}\in dom(w')\backslash dom(w)$ then $x\notin fv(e_1^{p_1})$
		\end{itemize}
		Then by \cref{def:fv}, we must also have
		\begin{itemize}
			\item if $x^{p_3}\in dom(w')\backslash dom(w)$ then $x\notin fv(e^{p'})$
		\end{itemize}

	\item[\runa{Ref write}] Here $e^{p'}=[e_1^{p_1}:=e_2^{p_2}]^{p'}$, where
		\begin{figure}[H]
			\setlength\tabcolsep{8pt}
			\begin{tabular}{l}
			\inference[]
				{env \vdash \left\langle e_1^{p_1},sto,w,p \right\rangle \rightarrow \left\langle \loc,sto',w',(L,V),p_1 \right\rangle &\\
				env \vdash \left\langle e_2^{p_2},sto',w',p_1 \right\rangle \rightarrow \left\langle v,sto'',w'',(L',V'),p_2 \right\rangle}
				{env\vdash \left\langle [e_1^{p_1}:=e_2^{p_2}]^{p'},sto,w,p \right\rangle \rightarrow \left\langle (),sto_3,w_3,(L,V),p' \right\rangle}
			\end{tabular}
		\end{figure}
		Where $sto_3=sto''[\loc\mapsto v]$ and $w_3=w''[\loc^{p_3}\mapsto(L',V')]$.
		By our induction hypothesis, we have:
		\begin{itemize}
			\item if $x^{p_3}\in dom(w')\backslash dom(w)$ then $x\notin fv(e_1^{p_1})$
			\item if $x^{p_3}\in dom(w'')\backslash dom(w')$ then $x\notin fv(e_2^{p_2})$
		\end{itemize}
		Then by \cref{def:fv}, we must also have
		\begin{itemize}
			\item if $x^{p_3}\in dom(w'')\backslash dom(w)$ then $x\notin fv(e^{p'})$
		\end{itemize}
	\end{description}
\end{proof}

\begin{lemma}[Strengthening]\label{lemma:Strength}
	If $\Gamma,x^{p'}:T';\Pi\vdash e^{p}:T$ and $x\notin fv(e^p)$, then $\Gamma;\Pi\vdash e^{p}:T$
\end{lemma}
\begin{proof}
	The proof proceeds by induction on the height of the derivation tree for $\Gamma;\Pi\vdash e^{p}:T$.
	\begin{description}
		\item[\runa{Cons}] Here $e^p=c^p$, where
		$$\Gamma,x^{p'}:T';\Pi\vdash c^p : (\emptyset,\emptyset)$$
		and $x\notin fv(c^p)$.
		We can thus conclude that:
		$$\Gamma;\Pi\vdash c^p : (\emptyset,\emptyset)$$

		\item[\runa{Var}] Here $e^p=y^p$, where
		$$\Gamma,x^{p'}:T';\Pi\vdash y^p : T$$
		and $x\notin fv(y^p)$.
		We can thus conclude that:
		$$\Gamma;\Pi\vdash y^p : T$$
	\end{description}
	Next, follows the induction step:

	\begin{description}
		\item[\runa{Abs}] Here $e^p=[\lambda\;y.e_1^{p_1}]^p$, where
		\begin{figure}[H]
			\setlength\tabcolsep{8pt}
			\begin{tabular}{l}
			\inference[]
				{\Gamma,y^{p''}:T_1;\Pi\vdash  e_1^{p_1}:T_2}
				{\Gamma,x^{p'}:T';\Pi\vdash  [\lambda\;y.e_1^{p_1}]^{p}:T_1\rightarrow T_2}\\[1cm]
			\end{tabular}
		\end{figure}
		and $x\notin fv([\lambda\;y.e_1^{p_1}]^p)$
		Then by \cref{def:fv} we know that $x\notin fv(e^{p})$.
		We can then conclude that $\Gamma;\Pi\vdash[\lambda\;y.e_1^{p_1}]^{p}:T_1\rightarrow T_2$.

		\item[\runa{App}] Here $e^p=[e_1^{p_1}\;e_2^{p_2}]^p$, where
		\begin{figure}[H]
			\setlength\tabcolsep{8pt}
			\begin{tabular}{l}
			\inference[]
				{
					\Gamma;\Pi\vdash e_1^{p_1}:T_1\rightarrow T_2 &\\
					\Gamma;\Pi\vdash e_2^{p_2}:T_1
				}
				{\Gamma,x^{p'}:T';\Pi\vdash [e_1^{p_1} \; e_2^{p_2}]^{p}:T_2}\\[1cm]
			\end{tabular}
		\end{figure}
		and $x\notin fv([e_1^{p_1} \; e_2^{p_2}]^p)$.
		By \cref{def:fv}, we then know:
		\begin{itemize}
			\item $x\notin fv(e_1^{p_1})$,
			\item $x\notin fv(e_2^{p_2})$
		\end{itemize}
		Where we can the conclude that $\Gamma;\Pi\vdash [e_1^{p_1} \; e_2^{p_2}]^{p}:T_2$.

		\item[\runa{Let 1}] Here $e^p=[\mbox{let}\;y\;e_1^{p_1}\;e_2^{p_2}]^p$, where
		\begin{figure}[H]
			\setlength\tabcolsep{8pt}
			\begin{tabular}{l}
			\inference[]
				{\Gamma;\Pi\vdash e_1^{p_1}:(\delta,\kappa) &\\
				\Gamma,y^p:(\delta,\kappa\cup\{x\});\Pi\vdash e_2^{p_2}:T_2}
				{\Gamma,x^{p'}:T';\Pi\vdash [\mbox{let}\; y \; e_1^{p_1} \; e_2^{p_2}]^{p}:T_2}
			\end{tabular}
		\end{figure}
		and $x\notin fv([\mbox{let}\; y \; e_1^{p_1} \; e_2^{p_2}]^p)$.
		By \cref{def:fv}, we then know:
		\begin{itemize}
			\item $x\notin fv(e_1^{p_1})$,
			\item $x\notin fv(e_2^{p_2})$
		\end{itemize}
		Where we can then conclude that $\Gamma;\Pi\vdash [\mbox{let}\; y \; e_1^{p_1} \; e_2^{p_2}]^{p}:T_2$.

		\item[\runa{Let 2}] Here $e^p=[\mbox{let}\;y\;e_1^{p_1}\;e_2^{p_2}]^p$, where
		\begin{figure}[H]
			\setlength\tabcolsep{8pt}
			\begin{tabular}{l}
			\inference[]
				{\Gamma;\Pi\vdash e_1^{p_1}:T_1 &\\
				\Gamma,y^p:T_1;\Pi\vdash e_2^{p_2}:T_2}
				{\Gamma,x^{p'}:T';\Pi\vdash [\mbox{let}\; y \; e_1^{p_1} \; e_2^{p_2}]^{p}:T_2}
			\end{tabular}
		\end{figure}
		and $x\notin fv([\mbox{let}\; y \; e_1^{p_1} \; e_2^{p_2}]^p)$.
		By \cref{def:fv}, we then know:
		\begin{itemize}
			\item $x\notin fv(e_1^{p_1})$,
			\item $x\notin fv(e_2^{p_2})$
		\end{itemize}
		Where we can then conclude that $\Gamma;\Pi\vdash [\mbox{let}\; y \; e_1^{p_1} \; e_2^{p_2}]^{p}:T_2$.
	\end{description}
\end{proof}

\subsection{Proof}
\begin{theorem}[Soundness of type system]
	Suppose $e^p$ is an occurrence where
	\begin{itemize}
		\item $\Gamma,\Pi\vdash e^p : T$, and 
		\item $env\vdash\left\langle e^p,w,sto,p'\right\rangle\rightarrow\left\langle v,w',sto',(L,V),p''\right\rangle$
	\end{itemize}
	and that $(w,env)\models\Gamma$.
	Then we have that:
	\begin{itemize}
		\item $\Gamma,\Pi\vdash v : T$
		\item $(w',env)\models\Gamma$
		\item $((L,V), w', env)\models (\Gamma,T)$
	\end{itemize}
\end{theorem}
\begin{proof}
	The proof proceeds by induction on the height for the derivation tree for $env\vdash\left\langle e^p,w,sto,p'\right\rangle\rightarrow\left\langle v,w,sto,L,V,p\right\rangle$ and $\Gamma,\Pi\vdash e^p:T$
	In the base case we have the \runa{Cons} and \runa{Var} rule:
	\begin{description}
		\item[\runa{Cons}] Here $e^p=c^p$, where
			\begin{itemize}
				\item $\Gamma,\Pi\vdash c^p : (\emptyset,\emptyset)$, and 
				\item $env\vdash\left\langle c^p,w,sto,p'\right\rangle\rightarrow\left\langle c,w',sto',(\emptyset,\emptyset),p\right\rangle$
			\end{itemize}
			and $(w,env)\models\Gamma$.
			This case is immediate, since there are no extensions to $w$, $env$, or $\Gamma$.
			The type also agrees with the dependency pair, where we then get:
			\begin{itemize}
				\item $\Gamma,\Pi\vdash c : (\emptyset,\emptyset)$
				\item $(w,env)\models\Gamma$
				\item $(env,(\emptyset,\emptyset))\models (\Gamma,(\emptyset,\emptyset))$
			\end{itemize}

		\item[\runa{Var}] Here $e^p=x^p$, where
			\begin{itemize}
				\item $\Gamma,\Pi\vdash x^p : T\sqcup (\{x^p\},\emptyset)$, and 
				\item $env\vdash\left\langle x^p,w,sto,p'\right\rangle\rightarrow\left\langle v,w',sto',(\emptyset,\emptyset),p\right\rangle$
			\end{itemize}
			and $(w,env)\models\Gamma$.
			This case is immediate, since there are no extensions to $w$, $env$, or $\Gamma$, where we then get:
			\begin{itemize}
				\item $\Gamma,\Pi\vdash v : T$
				\item $(w,env)\models\Gamma$
				\item $(env,(L,V)\models (\Gamma,T)$
			\end{itemize}
	\end{description}

	Next, follows the induction step:
	\begin{description}
		\item[\runa{Abs}] Here $e^p=[\lambda\;x\;e_1^{p'}]^p$, where
			\begin{itemize}
				\item $\Gamma,\Pi\vdash [\lambda\;x\;e^{p'}]^p : T_1\rightarrow T_2$, and 
				\item $env\vdash\left\langle [\lambda\;x\;e^{p'}]^p,w,sto,p'\right\rangle\rightarrow\left\langle \left\langle x,e^{p'},env\right\rangle,w',sto',(\emptyset,\emptyset),p\right\rangle$
			\end{itemize}
			and $(w,env)\models\Gamma$.

			As there is no extension to $w$ and $env$, and the dependency pair is $(\emptyset,\emptyset)$, we then get:
			\begin{itemize}
				\item $(w,env)\models\Gamma$
				\item $(env,(\emptyset,\emptyset)\models (\Gamma,T)$
			\end{itemize}
			The only step left is to type the value $v$.
			Since we know that the value $v$ is a closure, then the type rule we need it \runa{Closure}.
			We know that both the \runa{Abs} and \runa{Closure} type rules both have the abstraction type, and we know that the \runa{Abs} rule is of the form:
			\begin{figure}[H]
				\setlength\tabcolsep{8pt}
				\begin{tabular}{l}
					\InfName{Abs}\\[0.2cm]
						\inference[]
						{\Gamma,x^{p'}:T_1;\Pi\vdash  e^{p}:T_2}
						{\Gamma;\Pi\vdash  [\lambda\;x.e^{p}]^{p'}:T_1\rightarrow T_2}\\
				\end{tabular}
			\end{figure}
			And the \runa{Closure} rule is of the form:
			\begin{figure}[H]
				\setlength\tabcolsep{8pt}
				\begin{tabular}{l}
					\runa{Closure}\\[0.4cm]
						\inference[]
							{
								\Gamma;\Pi\vdash env \\
								\Gamma,x^{p}:T_1;\Pi\vdash e^{p'}:T_2
							}
							{\Gamma;\Pi\vdash \left\langle x^{p}, e^{p'}, env \right\rangle^{p''}:T_1'\rightarrow T_2'}\\
				\end{tabular}
			\end{figure}
			It is immediate to see that the type of the conclusion should be $T_1\rightarrow T_2$ and the type of the second premise should be $T_2$.
			Furthermore we also know that $x^{p'}=x^{p}$, and as such $T_1=T_1'$.
			Lastly is the first premise, $\Gamma;\Pi\vdash env$, which simple state that $\Gamma$ agrees with $env$, which we have from $(env,w)\models\Gamma$.

		\item[\runa{App}] Here $e^p=[e_1^{p'}\;e_2^{p''}]^p$, where
			\begin{itemize}
				\item $\Gamma,\Pi\vdash [e_1^{p'}\;e_2^{p''}]^p : T,$, and 
				\item $env\vdash\left\langle [e_1^{p'}\;e_2^{p''}]^p,w,sto,p'\right\rangle\rightarrow\left\langle v,w',sto',(L,V),p\right\rangle$
			\end{itemize}
			and $(w,env)\models\Gamma$.
			Where, by the \runa{App} semantics rule we have:
			\begin{figure}[H]
				\setlength\tabcolsep{8pt}
				\begin{tabular}{l}
					\InfName{App}\\[0.2cm]
					\inference[]
						{env \vdash \left\langle e_1^{p'},sto,w,p_3 \right\rangle \rightarrow \left\langle v',w_1, sto_1,(L_1,V_2),p' \right\rangle &\\
						env \vdash \left\langle e_2^{p''},w_1, sto_1,p' \right\rangle \rightarrow \left\langle v'',w_2, sto_2,(L_2,V_2),p'' \right\rangle &\\
						env[x\mapsto v''] \vdash \left\langle e^{p_1},w_3, sto_2, p'' \right\rangle \rightarrow \left\langle v,w', sto',(L_3,V_3),p_1 \right\rangle}
						{env\vdash \left\langle [e_1^{p'}\;e_2^{p''}]^{p},sto,w,p_3 \right\rangle \rightarrow \left\langle v,sto',w',(L,V),p \right\rangle}\\
					Where $v=\left\langle x,e^{p_1},env\right\rangle$, $w_3=w_2[x^{p''}\mapsto (L_2,V_2)]$, and \\
					$(L,V)=(L_1\cup L_3,V_1\cup V_3)$\\[1cm]
				\end{tabular}
			\end{figure}
			We need to show for: \runa{1} $(w',env)\models\Gamma$, \runa{2} $\Gamma;\Pi\vdash v:T$, and \runa{3} $((L,V),env)\models(\Gamma,T)$.
			\begin{description}
				\item[\runa{1}] From \cref{def:EnvAgree}, for environment agreement, we have two cases:
					\todo[inline]{This needs to be extended, such that the argument also holds for locations}
					\begin{itemize}
						\item If $x^{p_2}\in dom(w')\cap dom(w)$, then it follows the premises from \cref{def:EnvAgree}.
						\item If $x^{p_3}\in dom(w')\backslash dom(w)$, then from \cref{lemma:His}, we have $x\notin fv(e^{p})$ and as such $(w',env)\models\Gamma$ still holds since $x^{p_3}\notin dom(\Gamma)$.
					\end{itemize}
				\item[\runa{2}] Due to the type rule \runa{App}, we have:
					\begin{figure}[H]
						\setlength\tabcolsep{8pt}
						\begin{tabular}{l}
							\runa{App}\\[0.2cm]
								\inference[]
								{
									\Gamma;\Pi\vdash e_1^{p'}:T_1\rightarrow T &\\
									\Gamma;\Pi\vdash e_2^{p''}:T_1
								}
								{\Gamma;\Pi\vdash [e_1^{p'} \; e_2^{p''}]^{p}:T}\\
						\end{tabular}
					\end{figure}
					And per induction hypothesis, we have:
					\begin{description}
						\item[\runa{*1}] $\Gamma,\Pi\vdash v':T_1\rightarrow T$
						\item[\runa{*2}] $\Gamma,\Pi\vdash v'':T_1$
					\end{description}
					Where we then need to show that $\Gamma,\Pi\vdash v:T_2$ where $v=\left\langle x,e^{p_1},env\right\rangle$, which got concluded by \runa{Closure}:
					\begin{figure}[H]
						\setlength\tabcolsep{8pt}
						\begin{tabular}{l}
							\runa{Closure}\\[0.4cm]
							\inference[]
							{
								\Gamma;\Pi\vdash env \\
								\Gamma,x^{p'_1}:T_1;\Pi\vdash e^{p_1}:T
							}
							{\Gamma;\Pi\vdash \left\langle x^{p'_1}, e^{p_1}, env \right\rangle:T_1\rightarrow T}\\[1cm]
						\end{tabular}
					\end{figure}
					Since $\Gamma;\Pi\vdash env$, then we lastly also need to show that $\Gamma,x^{p'_1}:T_1;\Pi\vdash env[x\mapsto v]$.
					By the induction hypothesis, we have $\Gamma,x^{p'_1};\Pi\vdash v:T$, and by strengthening, \cref{lemma:Strength}, we have $\Gamma;\Pi\vdash v:T$, since $v$ is instantiated we know that $x\in dom(env)$ then $x\notin fv(v)$.
				\item[\runa{3}] $((L,V), w', env)\models T$
					By our induction hypothesis, we have:
					\begin{itemize}
						\item $((L_1,V_1),env,w_1)\models T_1\rightarrow T$,
						\item $((L_2,V_2),env,w_2)\models T_1$
					\end{itemize}
					To show that $((L,V),env,w')\models T$ we need to conclude the application.
					Since we have $\Gamma;\Pi\vdash v:T$ is concluded by \runa{Closure}, where the type got concluded by the second premise, we can then from our induction hypothesis get:
					$$((L_3,V_3),env,w')\models T$$
					Since the pair $(L,V)$ is $(L_1\cup L_3,V_1\cup V_3)$, we can then conclude that $((L,V),env,w')\models T$.
			\end{description}
			

		\item[\runa{Let}] Here $e^p=[\mbox{let}\;x\;e_1^{p'}\;e_2^{p''}]^p$, where
			\begin{itemize}
				\item $\Gamma,\Pi\vdash [\mbox{let}\;x\;e_1^{p'}\;e_2^{p''}]^p : T,$, and 
				\item $env\vdash\left\langle [\mbox{let}\;x\;e_1^{p'}\;e_2^{p''}]^p,w,sto,p'\right\rangle\rightarrow\left\langle v,w',sto',(L,V),p\right\rangle$
			\end{itemize}
			and $(w,env)\models\Gamma$.
			We must show that:
			\begin{itemize}
				\item $\Gamma,\Pi\vdash v : T$
				\item $(w',env)\models\Gamma$
				\item $(env,(L,V))\models (\Gamma,T)$
			\end{itemize}
			To do this, we need to analyse the premises, where in the semantics we have:
			\begin{figure}[H]
				\setlength\tabcolsep{8pt}
				\begin{tabular}{l}
					\InfName{Let}\\[0.2cm]
						\inference[]
						{env\vdash \left\langle e_1^{p'},sto,w,p_1 \right\rangle \rightarrow \left\langle v_1,sto'',w'',(L',V'),p' \right\rangle &\\
						env[x\mapsto v_1]\vdash \left\langle e_2^{p''},sto',w_3,p' \right\rangle \rightarrow \left\langle v,sto',w',(L,V),p'' \right\rangle}
						{env\vdash \left\langle [\mbox{let}\;x\;e_1^{p'}\;e_2^{p''}]^{p},sto,w,p_1 \right\rangle \rightarrow \left\langle v,sto',w',(L,V),p \right\rangle}\\
						Where $w_3=w''[x^{p'}\mapsto(L',V')]$\\
				\end{tabular}
			\end{figure}

			In the type system, we have to let rules, \runa{Let-1} and \runa{Let-2}, which differs in the typing of the first premise, namely when the value is a location.
			Thus, \runa{Let-1} is for when $v_1=\loc$ which corresponds when the type is $(\delta,\kappa)$ and $\kappa\neq\emptyset$, and \runa{Let-2} for other cases.

			When $v_1=\loc$, we have the following type rule
			\begin{figure}[H]
				\setlength\tabcolsep{8pt}
				\begin{tabular}{l}
					\runa{Let-1}\\[0.2cm]
						\inference[]
							{\Gamma;\Pi\vdash e_1^{p'}:(\delta,\kappa) &\\
							\Gamma,x^{p'}:(\delta,\kappa\cup\{x\});\Pi\vdash e_2^{p''}:T}
							{\Gamma;\Pi\vdash [\mbox{let}\; x \; e_1^{p'} \; e_2^{p''}]^{p}:T}\\[0.3cm]
						Where $\kappa\neq\emptyset$\\
				\end{tabular}
			\end{figure}
			For the first premise we can get $(w'',env)\models\Gamma$, as the dependency function is only extended.
			If we know $(env,(L',V'))\models(\Gamma,(\delta,\kappa))$, we can then then get $\Gamma;\Pi\vdash\loc:(\delta,\kappa)$.
			Since the type is $(\delta,\kappa)$, we know that the type is either from \runa{Var}, \runa{Ref}, or \runa{Ref read} in $e_1^{p''}$.

			In the second premise, we have $(w_3,env[x\mapsto v_1])\models\Gamma$, since $x^{p'}\notin dom(\Gamma)$.
			We can then also get $(w_3,env[x\mapsto v_1])\models(\Gamma,x^{p'}:(\delta,\kappa\cup\{x\}))$, since $(env,(L',V'))\models(\Gamma,(\delta,\kappa))$, and $x$ is and alias $\loc$.
			
			Lastly, we can the get:
			\begin{itemize}
				\item $\Gamma,\Pi\vdash v : T$,
				\item $(w',env)\models\Gamma$, and
				\item $(env,(L,V))\models (\Gamma,T)$
			\end{itemize}
			And the conclusion follows immediately.
			\bigskip

			When $v_1\neq\loc$, we have the following type rule:
			\begin{figure}[H]
				\setlength\tabcolsep{8pt}
				\begin{tabular}{l}
					\runa{Let-2}\\[0.2cm]
						\inference[]
							{\Gamma;\Pi\vdash e_1^{p'}:T_1 &\\
							\Gamma,x^{p'}:T_1;\Pi\vdash e_2^{p''}:T}
							{\Gamma;\Pi\vdash [\mbox{let}\; x \; e_1^{p'} \; e_2^{p''}]^{p}:T}\\
				\end{tabular}
			\end{figure}
			The proof follows similarly for the case where $v_1=\loc$.


		\item[\runa{Let rec}] Here $e^p=[\mbox{let rec}\;f\;e_1^{p'}\;e_2^{p''}]^p$

		\item[\runa{Case}] Here $e^p=$

		\item[\runa{Match}] Here $e^p=$

		\item[\runa{Ref}] Here $e^p=$

		\item[\runa{Ref read}] Here $e^p=$

		\item[\runa{Ref write}] Here $e^p=$
	\end{description}
\end{proof}
\end{document}
