\documentclass[../../master.tex]{subfiles}
\begin{document}
\begin{itemize}
	\item Introduction (what and how)
	\begin{itemize}
		\item What we will go through
		\item What do we mean with the system being sound
		\item How do we compare the system to the semantic (evaluation)
	\end{itemize}
	\item What should we know to prove the soundness of the liveness system
	\item Agreement of the different basis
	\begin{itemize}
		\item What does it mean to aggree
		\item How does $\Gamma$ and $\Upsilon$ aggree with $w$ and $env$?
		\item How does $\Pi$ aggree with the semantics
		\item How does the live type aggree with the semantics?
	\end{itemize}
	\item Soundness of liveness system (theorem
	\begin{itemize}
		\item What and why
		\item How is it good and what does it say
	\end{itemize}
	\item Rewrite the collection semantics
	\begin{itemize}
		\item The pair $(L,V)$ should contain all live locations and variables
	\end{itemize}
\end{itemize}

\section{Notations}
This section describes the notations used in the following sections.

\paragraph{Program point}
A program point is a labelling of program fragments, allowing to distinguish each expression, and subexpressions, in the program.
We define labelled program fragments as expressions and terms as unlabelled fragments.

\paragraph{Occurrence}
An occurrence is either a syntactical or semantical element, labelled with a program point.
Syntactical occurrences are elements named at some program point, expressed as $e^p$.
Semantical occurrences are elements applied at some program point, expressed as either $\loc^p$ or $x^p$ for location or variable occurrences, respectively.

\paragraph{Set notation}
Sets a defined as upper case letters.

\paragraph{Set occurrence}
When labelling a set with a program point, it indicates that the set contains occurrences, e.g. $V^p=\{x_1^{p_1},\cdots,x_n^{p_n}\}$ where $n$ is the number of elements in the set.

\section{Soundness}
In this section we will present the soundness of the liveness system, in respect to the collection semantics.
The soundness follows that if a variable is used in the evaluation of a program, then this variable needs to be deemed live in the liveness system.

To show the soundness of the liveness system, we need to define the agreement between the collection semantic and the liveness system.
This follows that we show how the basis', in the liveness system, agrees with the collection semantics, for a given term, and how the liveness type is a safe approximation of the used variable occurrences.

We introduced the dependency function $w$ as a function that maps variable and location occurrences to a set of variable and location occurrences.
In the liveness system, we defined the dependency of occurrences through the global $\Gamma$ basis.
Since the notion of locations are purely semantical, we instead have introduced the notion of abstract locations, as equivalence classes through the global basis $\Upsilon$.
Following this, we define the agreement between the basis' $\Gamma$ and $\Upsilon$ and the dependency function $w$.
The definition for the agreement of occurrence dependency also make use of an environment $env$, to lookup which variables are bound to locations.

\begin{definition}{[Agreement of the dependency of occurrences]}
	Let $w$ be the dependency function, $env$ be the environment, $\Gamma$ be the basis for variable occurrences, and $\Upsilon$ be the basis for abstract locations.
	We say that
	$$(w,env)\models(\Gamma,\Upsilon)$$
	Iff
	\begin{enumerate}
		\item $dom(env)\supseteq dom(\Upsilon)$
		\item $\forall x\in dom(\Upsilon). \exists \loc\in\Loc$ such that $env(x)=\loc$
		\item $\forall x^p\in dom(w).w(x^p)=(L^p,V^p)\Rightarrow \Gamma(x^p)=\delta\wedge V^p\subseteq\delta\wedge \Upsilon(x)=\kappa\wedge L^p\subseteq\kappa$
	\end{enumerate}
\end{definition}

The agreement for the dependency between occurrences follows that the dependencies in $w$ should also be defined by the pair $(\Gamma,\Upsilon)$
This is a sound assumption since both $\Gamma$ and $\Upsilon$ are both global while $w$ is local.
This, for one, means that $\Gamma$ and $\Upsilon$ can contain variable occurrences that is not defined in $w$.
For second, to be able to describe $w$ syntactical, we need both $\Gamma$ and $\Upsilon$ as they define variable occurrences and abstract locations, respectively.

For the pairs, $(w,env)$ and $(\Gamma,\Upsilon)$, to agree it follows that:
\begin{description}
	\item[(1)] Every variable bound by $\Upsilon$ should also be bound by $env$, since $env$ describes every variable and location occurrences, while $\Upsilon$ only defines abstract locations.
	\item[(2)] Every variable bound by $\Upsilon$ must be bound to a location in $env$.
	\item[(3)] For all variable occurrence, $x^p$, bound in $w$, the occurrence pair $(L^p,V^p)$ must agree with the dependency obtained from $\Gamma$ and the abstract location set obtained from $\Upsilon$.
		This means that the dependency from $\Gamma(x^p)=\delta$ must be a superset of $V^p$, and the abstract location set, obtained from $\Upsilon(x)=\kappa$ must be a superset of $L^p$.
\end{description}

Next follows the agreement betwen the approximated order of program points $\Pi$ and the dependency function $w$.
Here, it is assumed that the order between occurrences can be derived from the dependency function $w$.
This assumption is based on that a occurrences can only be bound to occurrences at an earlier program point.


%The agreement of the order of program points is defined from $w$ and $\Pi$.
%Here, it is considered that the order for program points can be derived from the dependency function $w$, where $\forall u^p\in dom(w)$, where $u\in\Loc\cup\Var$, then all variables $u$ is dependent on is evaluated before $u$.

\begin{definition}{[Agreement of program order]}
	Let $w$ be the dependency function and $\Pi$ be an approximation of the order of program points, then we say that
	$$w\models\Pi$$
	Iff $\forall u^p\in dom(w).w(u^p)=(L^p,V^p)$, where $u\in(\Var\cup\Loc)$, then
	\begin{enumerate}
		\item if $\loc^q\in L^p$ then $q\sqsubseteq p$
		\item if $y^q\in V^p$ then $q\sqsubseteq p$
	\end{enumerate}
	\iffalse
	\begin{itemize}
		\item $(p,q)\in\Pi \Rightarrow \exists x,y. x^p\in dom(w) \wedge y^q\in dom(w)$
		\item $w(u^p)=(L^p,V^p)$, where $u\in\Var\cup\Loc)$, then
		\begin{enumerate}
			\item if $\loc^q\in L^p$ then $q\sqsubseteq p$
			\item if $y^q\in V^p$ then $q\sqsubseteq p$
		\end{enumerate}
	\end{itemize}
	\fi
\end{definition}

The agreement for the order of program points between the basis $\Pi$ and the dependency function $w$, is based on that the order derived from $w$ is in agreement with $\Pi$.
This agreement is good, since the program point order derived from the dependency function $w$ is also the same in the basis $\Pi$.
%This agreement is good, the dependency function describes dependencies and program point order, for a given evaluation, and the order
\bigskip

The last agreement can now be defined, for the liveness type.
This agreement uses a notation to retrive one component from a occurrence pair, defined as follows:

%With the agreement between the three basis' and the collection semantic, we can now define the agreement for the liveness type $(\delta,\kappa)$.
%In the agreement for the liveness type, we use two notations for restricting on the dependency pair, to retrive one component, defined as follows:
\begin{itemize}
	\item $(L,V)|_L \overset{def}{=} L$
	\item $(L,V)|_V \overset{def}{=} V$
\end{itemize}

\todo[inline]{Something is clearly wrong with this definition, since $w$ only describes the dependencies from a specific branch, and $\delta$ is an overapproximation (i.e. $\delta$ can include variable occurrences from other branches)}
\begin{definition}{[Agreement of liveness type and the dependency function]}
	Let $w$ be the dependency function, $env$ be the environment, $\delta$ be a set of possible live variables, $\kappa$ be the abstract locations.
	We say that:
	$$(w,env)\models(\delta,\kappa)$$
	Iff
	\begin{enumerate}
		\item $dom(w)\supseteq\delta$
		\item $dom(w)\supseteq\kappa$
		\item $\forall x^p\in dom(w)\cap\delta. w(x^p)|_V\supseteq\delta$, and
		\item $\forall \loc^p.\exists x\in\kappa. env(x)=\loc \Rightarrow w(\loc^p)|_L\supseteq\kappa$
	\end{enumerate}
\end{definition}

The agreement between the liveness type and the dependency function.

\begin{theorem}{[Soundness of type system]}
	If $(w,env)\models(\delta,\kappa)$ and $(w,env)\models(\Gamma,\Upsilon)$ holds in
	\begin{itemize}
		\item $\Gamma,\Upsilon,\Pi\vdash e^p : (\delta,\kappa)$, and 
		\item $env\vdash\left\langle e^p,w,sto,p'\right\rangle\rightarrow\left\langle v,w',sto',L,V,p''\right\rangle$.
	\end{itemize}
	Then $(w',env)\models(\Gamma,\Upsilon)$, $(w',env)\models(\delta,\kappa)$, and $\forall y^p\in V.w(y^p)=(L,V)\wedge V\subseteq \delta$ also holds.
\end{theorem}

\begin{proof}
	In the base case, consider the following rules, for $CONST$ and $VAR$:
	\begin{description}
		\item[$\lbrack CONST \rbrack$] For the $CONST$ rule, we know the following:
			\begin{itemize}
				\item $\Gamma,\Upsilon,\Pi\vdash c^p :(\emptyset,\emptyset)$,
				\item $env\vdash\left\langle c^p,w,sto,p'\right\rangle\rightarrow\left\langle c,w,sto,\emptyset,\emptyset,p\right\rangle$,
				\item $(w,env)\models(\Gamma,\Upsilon)$, and
				\item $(w,env)\models(\delta,\kappa)$
			\end{itemize}
			Since no update to $w$ happens in the evaluation for the $CONST$ rules, we know that $(w',env)\models(\Gamma,\Upsilon)$ and $(w',env)\models(\delta,\kappa)$, where $w'=w$, still holds.
			And since $V=\emptyset$ in the $CONST$ rule, we also know that the third criteria holds, since $\emptyset\subseteq\emptyset$
		\item[$\lbrack VAR \rbrack$]
	\end{description}
	Inductive step
	\begin{description}
		\item[] ss
	\end{description}
\end{proof}

\iffalse
We have shown the liveness system, in section x, and the collection semantics, in section y.
This section will introduce the soundness for the liveness analysis, that is, it over approximate the live variables.
To show this, we need to show that the variables used in a potential evaluation in the collection semantics, the liveness system also deem those variables are live.

The liveness system uses some global information, given by the three basis, $\Gamma$, $\Upsilon$, and $\Pi$.
$\Gamma$ can be described as the abstract dependency function, $\Upsilon$ describes the abstract locations, and $\Pi$ describes the approximation of program points.

\begin{definition}{[Soundness of $\Gamma$ and $\Upsilon$]}
	Let $(w,env)$ be the dependency function and environment, and $(\Gamma, \Upsilon)$ be the basis pair for variable occurrences and abstract locations.
	We write:
	$$(w,env)\models(\Gamma,\Upsilon)$$
	if the following holds:
	\begin{itemize}
		\item $\forall x^p\in dom(w).\Gamma(x^p)=V_1^\loc \wedge (w^p)=(L^\loc,V^\loc)\wedge V^\loc\subseteq V_1^\loc$
		\item $\forall x,y\in\Var. env(x)=env(y)=\loc\Rightarrow x\sim_\Upsilon y$
	\end{itemize}
\end{definition}

\begin{lemma}{[Soundness for typesystem]}
kal kun tilmelde dig de f	The type system is sound if, given $(w,env)\models(\Gamma,\Upsilon)$ for
	\begin{itemize}
		\item $\Gamma,\Upsilon,\Pi\vdash e^p:(\delta,\kappa)$, and
		\item $env\vdash\left\langle e^p,sto,w,p'\right\rangle\rightarrow\left\langle v,sto',w',L,V,p''\right\rangle$
	\end{itemize}
	Then $(w',env)\models(\Gamma,\Upsilon)$ still holds and $p'\sqsubseteq p\in\Pi$.
\end{lemma}

\begin{definition}{[Soundness of $\Gamma$]}
	Let $(w,env)$ be the dependency function and environment, and $\Gamma$ be basis for variable occurrences, then we have that:
	$$(w,env)\models\Gamma$$
\end{definition}

\begin{definition}{[Soundness of $\Upsilon$]}
	Let $(env,sto)$ be the environment and store, and $\Upsilon$ be basis for abstract locations, then we have that:
	$$(env,sto)\models\Upsilon$$
	Iff $\forall x,y\in\Var.env(x)=env(y)=\loc\Rightarrow x\sim_\Upsilon y$
\end{definition}
Where the notation $x\sim_\Upsilon y$ is defined as:
$$x\sim_\Upsilon y \mbox{ if } \Upsilon(x)=\Upsilon(y)$$

\begin{definition}{[approximation of $\delta$]}
	Let $(L,V,w)$ be the location occurrences, the variable occurrences, and $w$ be the dependency function, respectively.
	Let $\delta$ be the set of live variables, then:
	$$(L,V,w,env)\models\delta$$
	Iff:
	\begin{itemize}
		\item $V\subseteq \delta$
		\item $\forall x\in V$ and $\forall p.w(x^p)=(L'',V'')$, where $(L'',V'',w)\models\delta$
		\item $\forall \loc\in L$ and $\forall p.w(\loc^p)=(L',V')$, where $(L',V',w)\models\delta$
		\item $\forall \loc\in L$ and $\forall x.env(x)=\loc$, where $x\in\delta$ and $(\emptyset,\{x\},w)\models\delta$
	\end{itemize}
\end{definition}

\begin{definition}{[Soundness for typesystem]}
	Let $\Gamma$ be the basis of variable occurrences, $\Upsilon$ be the basis of abstract locations, and $\Pi$ be the approximation of program points.
	Let $env$ be the environment and $sto$ be the store.
	Then, if $(w,env)\models\Gamma$, $(env,sto)\models\Upsilon$, and $\Gamma,\Upsilon,\Pi\vdash e^p:(\delta,\kappa)$, and $env\vdash\left\langle e^p,sto,w,p'\right\rangle\rightarrow\left\langle v,sto',w',L,V,p''\right\rangle$.
	Then $(w',env)\models\Gamma$, $(env,sto')\models\Upsilon$, $p'\sqsubseteq_\Pi p$, and $(L,V,w')\models\delta$ holds.
\end{definition}

\begin{proof}
	We prove the soundness of the typesystem by induction. In the base cases we consider the rules for $[Const]$, $[Var]$, $[Abs]$.
	First, consider the $[Const]$ rule.
	Since we have that $env\vdash\left\langle c^{p'},sto,w,p\right\rangle\rightarrow\left\langle c,sto,w,\emptyset,\emptyset,p'\right\rangle$ and $\Gamma,\Upsilon,\Pi\vdash e^{p'}:(\emptyset,\emptyset)$ it is immediate to see that, if $(w,env)\models\Gamma$, $(env,sto)\models\Upsilon$, and $\Gamma,\Upsilon,\Pi\vdash e^p:(\delta,\kappa)$ holds, then $(w,env)\models\Gamma$, $(env,sto)\models\Upsilon$, and $\Gamma,\Upsilon,\Pi\vdash e^p:(\delta,\kappa)$ also holds after evaluating a constant.
	Since there is no variable or location occurrences, and $L=V=\delta=\emptyset$ it is immediate that $(\emptyset,\emptyset,w,env)\models\emptyset$ holds. 

	Second, consider the $[Var]$ rule.
	From $env\vdash\left\langle x^{p'},sto,w,p\right\rangle\rightarrow\left\langle v,sto',w',\emptyset,\{x^{p'}\},p'\right\rangle$ and $\Gamma,\Upsilon,\Pi\vdash e^{p'}:(\delta\cup\{x^{p'}\},\kappa)$, after the evaluation it is immediate to see $(w',env)\models\Gamma$ and $(env,sto')\models\Upsilon$ still holds.
	Since $\Pi$ is and approximation of program points, then it is also immediate to see that $p\subseteq_\Pi p'$ and $p'\subseteq_\Pi p''$ holds.
	Lastly, we have $\delta$.
	Firstly it is immediate to see that $x^{p'}\in\delta\cup\{x^{p'}\}$.
	Secondly, since $(w,env)\models\Gamma$ holds, it is also immediate to see that $(\emptyset,V,w,env)\models\delta\cup\{x^{p'}\}$
	%it is immediate to see that $V\subseteq\delta\cup\{x^{p'}\}$


	Third, consider the $[Abs]$ rule.
\end{proof}
\fi
\end{document}
