\documentclass[../../master.tex]{subfiles}
\begin{document}
\section{Type system for data-flow analysis}\label{sec:TypeSys}
This section will introduce the type system for the data-flow analysis on the language presented in \cref{sec:lang}.
The type system presented in this section is a type checker for global analysis of occurrences.
The type checker assign types, presented in \cref{sec:types}, to occurrences given the basis, which will be presented in \cref{sec:basis}, and using the type assignment, which is presented in \cref{sec:Judge}.
Since the language contains local information as bindings, and global information as locations, the type checker should also reflect both.
Since locations are a semantic notation, we will present internal variables to represent locations in the type system, as $\nu x,\; \nu y\in\cat{IVar}$.

Since locations, and internal variables, are not always bound to a variable, and not all aliases are used, the alias analysis is done by only typing the information used for an occurrences.
Here, we are going to introduce the basis for aliasing, that partition all variables and internal variables.
As such, it is possible to analyse from the type information for an occurrence, which aliases are actually used.
As an example, consider the following occurrence:

We also impose some restrictions on the type system, which are known to be safe.
The first restrictions we impose are for that references cannot be bound to abstractions and the second is that abstractions can at most be used once, since the type system presented here does not allow for polymorphism.

\subsection{Types}\label{sec:types}
We denote the set of types as \cat{Types}, which are given by the following formation rules:

$$T::=(\delta,\kappa)\mid T_1 \rightarrow T_2$$

Here, we introduce two types, the base type $(\delta,\kappa)$ and the function type $T_1 \rightarrow T_2$.
The idea is that an occurrence have the abstraction type if it represents a function that takes an argument of type $T_1$ and returns a type of $T_2$.
The base type represent all other values, where $\delta$ represent the set of occurrences used to evaluate an occurrence and $\kappa$ represent the set of alias information.
Here, if an occurrence have a type containing alias information, then it represent a location, where if the occurrence have a base type with alias information, then the occurrence represents a location, 
and if the occurrence have the abstraction type where either $T_1$ or $T_2$ have are base types with alias information then the abstraction either takes a location as input or produces a location.

\begin{example}[]
Consider the following occurrence:
\begin{lstlisting}[language=Caml, mathescape=true]
(let x (3$^1$)$^2$ (let y (ref x$^3$)$^4$ (!y)))
\end{lstlisting}
Here, we can give $x$ the type $(\emptyset,\emptyset)$ as $x$ is bound to a constant and there a no variables or internal variables used.
$y$ can then be given the type $(\emptyset,\{y,\nu y\})$, as the reference construct $ref$ creates a new reference, which $y$ is then an alias to, e.g., $y$ is bound to a location.
Her $\nu y$ represents the reference from $ref$, and can thus be given the type $(\{x^3\},\emptyset)$, where $\nu y$ is bound to a constant, because of $x$, but the occurrence $x^3$ were used, so it should be part of the set of occurrences $\delta$.
\end{example}

Since the type system approximates the data used for terms, we introduce two unions.
The first union is a simple union that expects the types to be similar, that is, only the base types are allowed to be different.
\begin{definition}[Type union]
	Let $T_1$ and $T_2$ be two types, then the type union, $\cup$, are as follows:
	\begin{equation*}
		T_1\cup T_2=
		\left\{\begin{matrix}
			\mbox{If } \; T_1=(\delta,\kappa) \;\mbox{and}\; T_2=(\delta',\kappa')  & \mbox{then} \; (\delta\cup\delta',\kappa\cup\kappa')\\
			\mbox{else if } \; T_1=T_1'\rightarrow T_1''\;\mbox{and}\;T_2=T_2'\rightarrow T_2'' & \mbox{then} \; (T_1'\cup T_2')\rightarrow (T_1''\cup T_2'')
		\end{matrix}\right.
	\end{equation*}
\end{definition}

The second type union, is to add additional type information to an arbitrary type.
This type union is used to add an occurrence to a type, e.g., in the \runa{Var} rule where the variable occurrence needs to be added to the type of that variable.
\begin{definition}[Base type union]
	Let $T$ be an type and $(\delta,\kappa)$ be a base type, then the union of these are as follows:
	\begin{equation*}
		T\sqcup (\delta,\kappa)=
		\left\{\begin{matrix}
			\mbox{If } \; T=(\delta',\kappa')  & \mbox{then} \; (\delta\cup\delta',\kappa\cup\kappa')\\
			\mbox{else if } \; T=T_1\rightarrow T_2 & \mbox{then} \; T_1\rightarrow(T_2\sqcup (\delta,\kappa))
		\end{matrix}\right.
	\end{equation*}
\end{definition}

\subsection{Basis and type environment}\label{sec:basis}
Next, we will present the basis and type environment for the type system.
The basis we are presenting here are assumptions used by the type checker in addition to the assignment of types, which are presented in \cref{sec:Judge}, where we are going to present a type base for aliasing and an approximated order of program points.

We will also introduce the type environment, which are similar to the environment and store used in the semantics, as the type environment keeps track of the type of variables and internal variables.
As such, the type environment is also a approximation of the dependency function, as the purpose of the type system is to collect information about which occurrences are used.

Similar to the lookup of the greatest binding for the dependency function, we are going to introduce an instantiation of the function from \cref{def:GBind} for the type environment in respect to the basis for approximated order of program points.
\bigskip

We will then introduce the type base for aliasing, as a partition of variables and internal variables used in an occurrence.

\begin{definition}[Type Base for aliasing]
	For an occurrence $o$, let $var$ be the set of all variables and $ivar$ be the set of all internal variables in $o$.
	The type base $\kappa^0=\{\kappa^0_1,\cdots,\kappa^0_n\}$ is then a partition of $var\cup ivar$, where $\kappa_i^0\cap\kappa_j^0=\emptyset$ for all $i\neq j$.
\end{definition}

The idea behind the base for type alias $\kappa_0$ is to make a partition of the variables and internal variables used in an occurrence.
As such multiple variables can only belong to the same element $\kappa_0^i\in\kappa_0$, if there also exists an internal variable in $\kappa_0^i$.

\begin{definition}[Approximated order of program points]
	An approximated order of program points $\Pi$ is a pair, such that: 
	$$\Pi=(\cat{P},\sqsubseteq_\Pi)$$
	where
	\begin{itemize}
		\item \cat{P} is the set of program points in an occurrence,
		\item $\sqsubseteq_\Pi\subseteq\cat{P}\times\cat{P}$, where
		\begin{itemize}
			\item $\sqsubseteq_\Pi$ is transitive, such that if $p\sqsubseteq_\Pi p'$ and $p'\sqsubseteq_\Pi p''$ then $p\sqsubseteq_\Pi p''$.
		\end{itemize}
	\end{itemize}
\end{definition}

The approximated order of program points is an assumption about the order for program points, as such, this approximation should be an approximation of the order that that can be derived from the semantics presented in \cref{sec:sem}.

\begin{definition}[Partial order of $\Pi$]
	Let $\Pi=(\cat{P},\sqsubseteq_\Pi)$ be an approximated order of program points.
	We say that $\Pi$ is a partial order if $\sqsubseteq_\Pi$ is a partial order.
\end{definition}
\bigskip

Next, we will introduce the type environment:
\begin{definition}[Type Environment]
	A type environment $\Gamma$ is a partial function $\Gamma:\cat{Var}^P\cup\cat{IVar}^P\rightharpoonup\cat{Types}$
\end{definition}

\begin{definition}[Updating Type Environments]
	Let $\Gamma$ be a type environment, we write $\Gamma[u^p:T]$, for an occurrence $u^p$, to denote the type environment $\Gamma'$ where:
	\begin{align*}
		\Gamma'(y^{p'})=
		\left\{\begin{matrix}
			\Gamma(y^{p'}) & \mbox{if}\;y^{p'}\neq u^{p}\\\	 
			T & \mbox{if}\;y^{p'}=u^{p}
		\end{matrix}\right.
	\end{align*}
\end{definition}

Similar to the lookup of dependencies in the semantics, we need to similarly define how to lookup in the type environment.
As the type environment contains both local information, for local declarations, and global information, for references, we need to define how to lookup both.

For local information we introduce, similarly to lookup in the dependency function, and instantiation of the function presented in \cref{def:GBind}.
The lookup is for information in the type environment, while the relation between program points is defined in the basis for approximated program points.

\begin{definition}[]\label{def:GBindPi}
	Let $u\in \cat{Var}\cup\cat{IVar}$, be either a variable or internal variable, $\Gamma$ be a type environment, and $\Pi$ be the approximated order of program points that is a partial order, then $uf_{\sqsubseteq_\Pi}i$ is given by:
	$$uf_{\sqsubseteq_\Pi}(u,\Gamma)=\inf\{u^p\in dom(\Gamma)\mid u^q\in dom(\Gamma).q\sqsubseteq_\Pi p\}$$
\end{definition}

The lookup for global information needs to be handled differently as the language contains pattern matching, as such, the language can contain different path of evaluation (where each pattern in the pattern matching construct introduces a new path).
To handle the lookup of global information, we will first introduce the notion of $p$-chains as chains of program points with respect to the approximated order of program points, where the maximal program points is $p$.
The idea of these $p$-chains is to describe the history behind an occurrence $u^p$, and thus also describe the occurrences that $u^p$ depends on.

\begin{definition}[$p$-chains]
	Let $\Pi$ be an approximated order of program points, that is a partial order, and $p$ be a program point.
	We then say that a $p$-chain, denoted as $\Pi_p^{*}$, is a total order where the maximal element is $p$ with respect to the order $\Pi$.
	As such, the order $\Pi_p^{*}$ does not contain any pairs $(p,q)\in\sqsubseteq_\Pi$, where $p\neq q$, then $(p,q)\notin\sqsubseteq_{\Pi_p^{*}}$.
\end{definition}

We also denote $\Pi_p^{*}\in\Pi$, if the $p$-chain $\Pi_p^{*}$ can be be derived from $\Pi$.

Since there can exists multiple paths in an occurrence, we define the set of all $p$-chains as follows:

\begin{definition}[]
	Let $\Pi$ be an approximated order of program points and $p$ be a program point.
	We say that $\Upsilon_p$ is the set of all $p$-chains in $\Pi$.
\end{definition}

Since $\Upsilon_p$ contains all maximal $p$-chains in an approximated order of program points $\Pi$, with $p$ as the maximal element, we can then define the function to lookup all greatest element less than or equal to $p$.

\begin{definition}[]\label{def:GBindUps}
	Let $u\in \cat{Var}\cup\cat{IVar}$, be either a variable or internal variable, $\Gamma$ be a type environment, and $\Upsilon_p$ be a set of $p$-chains, then $uf_{Upsilon_p}$ is given by:
	$$uf_{\Upsilon_p}(u,\Gamma)=\bigcup_{\Pi_p^{*}\in\Upsilon_p}uf_{\Pi_p^{*}}(u,\Gamma)$$
\end{definition}

The function, defined in \cref{def:GBindUps}, takes the union of the greatest binding of each element using the function defined in \cref{def:GBindPi}.

\subsection{Judgement}\label{sec:Judge}
We will now present the judgement and type rules for the language, that is, how we assign types to terms.

The type judgement is defined as:
$$\Gamma,\Pi\vdash e^p: T$$
And should be read as, the occurrence $e^p$ has type $T$, given the dependency bindings $\Gamma$ and the approximated order of program points $\Pi$.

A highlight of type rules can be found in \cref{fig:TypeSys}, and all type rules can be found in \cref{App:TypeSys}.

\begin{description}
	\item[\runa{T-Const}] rule, for occurrence $c^p$, is the simplest type rule, as there is nothing to track for constants.

	\item[\runa{T-Var}] rule, for occurrence $x^p$, looks up the type in the environment, by finding the greatest binding using \cref{def:GBindPi}, and add the occurrence $x^p$ to the type.

	\item[\runa{T-Let-1}] rule, for occurrence $[\mbox{let}\;x\;e_1^{p_1}\;e_2^{p_2}]^p$, creates a local binding for a variable, with the type of $e_1^{p_1}$ that can be used in $e_2^{p_2}$.
		The \runa{T-Let-1} rule assumes that the type of $e_1^{p_1}$ is a base type with alias information, i.e., $\kappa\neq\emptyset$.
		If this is the case, then $e_1^{p_1}$ must evaluate to a location, in the semantics.
		The other cases, when $e_1^{p_1}$ is not a base type with alias information, are handled by the \runa{T-Let-2} rule.

	\item[\runa{T-Case}] rule, for occurrence $[\mbox{case}\;e^{p}\;tilde{\pi}\;tilde{o}]^{p'}$, is an over-approximation of all cases in the pattern matching expression, by taking an union of the type of each case.
		Since the type of $e^p$ is used to evaluate the pattern matching, we also add this type to the type of the pattern matching.


	\item[\runa{T-Ref-read}] rule, for occurrence $[!e^{p}]^{p'}$, is used to retrieve the type of references, where $e^p$ must thus be a base type with alias information.
		Since the type system is an over-approximation, there can be multiple internal variables in $\kappa$ and multiple occurrences we need to read from.
		As such we need to lookup for all internal variables and also possible for multiple program points.
		This is due to there can be multiple writes to the references, depending on which case in the pattern matching is evaluated, we need to consider all paths, as such we use the $\Lambda_\Upsilon$ function.
\end{description}

\begin{figure}[H]
	\setlength\tabcolsep{8pt}
	\begin{tabular}{l}
		\runa{T-Const}\\[0.2cm]
	\inference[]{}
	{\Gamma,\Pi\vdash  c^{p}:(\emptyset,\emptyset)}
\\[0.7cm]
		\runa{T-Var}\\[0.2cm]
	\inference[]{}
	{\Gamma,\Pi \vdash x^p:T \sqcup (\{x^p\},\emptyset)}\\[0.3cm]
	$x^{p'}=uf_{ \sqsubseteq_\Pi}(x,\Gamma)$, and $\Gamma(x^{p'})=T$
\\[0.7cm]
		\runa{T-Let-1}\\[0.2cm]
	\inference[]
	{
		\Gamma,\Pi\vdash e_1^{p_1}:(\delta,\kappa) &\\
		\Gamma',\Pi\vdash e_2^{p_2}:T_2
	}
	{\Gamma,\Pi\vdash [\mbox{let}\; x \; e_1^{p_1} \; e_2^{p_2}]^{p}:T_2}\\[0.3cm]
	Where $\Gamma'=\Gamma[x^{p}:(\delta,\kappa\cup \{x\})]$ and $\kappa\neq\emptyset$
\\[0.7cm]
		\runa{T-Case}\\[0.2cm]
	\inference[]
	{
		\Gamma,\Pi\vdash e^{p}:(\delta,\kappa) &\\
		\Gamma',\Pi\vdash e_i^{p_i}:T_i\;\;\;(1\leq i\leq|\tilde{\pi}|)
	}
	{\Gamma,\Pi\vdash [\mbox{case}\;e^{p}\;\tilde{\pi}\;\tilde{o}]^{p'}:T\sqcup(\delta,\kappa)}\\[0.3cm]
	Where $e_i^{p_i}\in\tilde{o}$ and $s_i\in\tilde{\pi}$ $T=\bigcup_{i=1}^{|\tilde{\pi}|}T_i$, and\\
	$\Gamma'=\Gamma[x^p:(\delta,\kappa)]$ if $s_i=x$
\\[0.7cm]
		\runa{T-Ref-read}\\[0.2cm]
	\inference[]
	{\Gamma,\Pi\vdash  e^{p}:(\delta,\kappa)}
	{\Gamma,\Pi\vdash [!e^{p}]^{p'}:T\sqcup(\delta\cup\delta',\emptyset)}\\
	Where $\kappa\neq\emptyset$, $\delta'=\{\nu x^{p'}\mid\nu x\in\kappa\}$, $\nu x_1,\cdots,\nu x_n\in\kappa$.\\ 
	$\{\nu x_1^{p_1},\cdots,\nu x_1^{p_m}\}=uf_{\Upsilon_{p'}}(\nu x_1,\Gamma),\cdots,\{\nu x_n^{p_1'},\cdots,\nu x_n^{p_s'}\}=uf_{\Upsilon_{p'}}(\nu x_n,\Gamma)$, and\\
	$T=\Gamma(\nu x_1^{p_1})\cup\cdots\cup\Gamma(\nu x_1^{p_m})\cup\cdots\cup\Gamma(\nu x_n^{p_1'})\cup\cdots\cup\Gamma(\nu x_n^{p_s'})$
\\[0.7cm]
	\end{tabular}
	\caption{Selected rules from the type system}
	\label{fig:TypeSys}
\end{figure}

\begin{example}[Data-flow for abstractions]
	Consider the following occurrence for application:
	\begin{lstlisting}[language=Caml, mathescape=true]
		(($\lambda$ y.(PLUS 3$^1$ y$^2$)$^3$)$^4$ 5$^5$)$^6$
	\end{lstlisting}
	The derivation tree for the occurrence can be found in \cref{FigEx.TAbs}.
	Here, we show two application \runa{T-App} and \runa{T-App-const}, where we create an abstraction for adding the constant $3$ to the argument of the abstraction, and applying the constant $5$ to the abstraction.

	When typing the abstraction, we need too make an assumption about the parameter $y$.
	As we are applying a constant to the argument, we can make an assumption that the type of the parameter should be $(\emptyset,\emptyset)$.

	Based on this assumption for the type, we can type the body of the abstraction.
	As the body is \runa{T-App-const} we take a union of the types
\end{example}

\begin{example}[Data-flow for references]
	Consider the following occurrence for application:
	\begin{lstlisting}[language=Caml, mathescape=true]
		(let x (ref 1$^1$)$^2$ (let y (x$^3$) (!x$^4$)$^5$)$^6$)$^7$
	\end{lstlisting}
	The derivation tree for the occurrence can be found in \cref{FigEx.TRef}.
	Here, we show the typing of references where we create a reference and create 2 aliases for it before reading from the reference.
	
	When typing the reference, it modifies the base type $\Gamma$ with a new internal variable.
	From the type information, it is clear the only the variable $x$ is used and the internal variable $\nu x$ is used.
\end{example}

\begin{landscape}
\subfile{../examples/DFAbsT.tex}
\end{landscape}
\end{document}
