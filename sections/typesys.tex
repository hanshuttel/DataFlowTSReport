\documentclass[../../master.tex]{subfiles}
\begin{document}
\section{type system}
This section will introduce the liveness type system, for the language presented in \cref{sec:lang}.
This type system will focus on the liveness property for variable occurrences, to describe which and where variables are used.

\subsubsection{Variables}
The language presented uses variables in different areas, where we have classic placeholder variables from the classical $\lambda$-calculus, in the form of abstractions, and also imperative variables.
The imperative variables are in the form of bindings, which are immutable by default, but can be bound to a location by referencing.

In the type system we differentiate between these variable types, by letting $\Var$, $\IVar$, and $\AVar$ be the categories for placeholder variables, imperative variables, and internal variables, respectively.
Note that, since the type system only knows about the syntax, and locations are purely semantical which does not need to be bound to a variable, we introduce the notation of internal variables.
These variables are denoted as $\nu x\in\AVar$ to differentiate from other variables.
We further denote $Id=\Var\cup\IVar\cup\AVar$ as the category of identifiers.

\subsubsection{Types}
The syntactic category of types $TYPES$ is ranged over by $T$ and is given by the following formation rules:

$$T::=(\delta,\kappa)\mid T_1 \rightarrow T_2$$

As can be seen by the formation rules, there are two ways of describing a type, a base type and a function type.
The base type is a pair containing the set of variable occurrences $\delta$, and a set of variables $\kappa$.
Here, $\delta$ is used to describe a set of live variable occurrences and $\kappa$ is a set of abstract locations.
The second type is the function type $T_1\rightarrow T_2$ that represents a function where the input has the type $T_1$ and returns a value of type $T_2$.

\subsubsection{Bases}
As also described earlier, this type system keeps track of the abstract liveness type, that keeps track of live variable occurrences.
In the type system we uses the base $\Gamma$ to track the type of identifier occurrences, $\Upsilon$ to track variables that share an abstract location, and $\Pi$ as an approximation of program points.

\begin{definition}{[Type base $\Gamma$]}
	The type base $\Gamma$ is a partial function $\Gamma:Id^P\rightharpoonup TYPES$
\end{definition}

A lookup in the type base $\Gamma$ is of the form $\Gamma(x^p)=T$ for placeholder and imperative variables, and $\Gamma(\nu x^p)=T$ for internal variables.

\begin{definition}{[Extending $\Gamma$]}
	We write $\Gamma,x^p:T$ for the type base $\Gamma'$ defined by:
	\begin{align*}
		\Gamma'(y^{p'})=
		\left\{\begin{matrix}
			\Gamma(x^p) & \mbox{if}\;y^{p'} \neq x^{p}\\ 
			T & \mbox{if}\;y^{p'} = x^{p}
		\end{matrix}\right.
	\end{align*}
\end{definition}
The extension of $\Gamma$ for internal variables can be similarly defined.
\bigskip

\begin{definition}{[Abstract location base $\Upsilon$]}
	The base for abstract locations $\Upsilon$ is a set of sets, $\Upsilon=\Pow((\IVar\cup\AVar))$
\end{definition}
We denote $\Upsilon(x)=\kappa$ as a lookup of all sets in $\Upsilon$ containing the variable $x$.

\begin{definition}{[Extending $\Upsilon$]}
	We write $\Upsilon,\kappa$ for $\Upsilon'$ defined by:
	\begin{align*}
		\Upsilon'(y)=\{[\tilde{x}]\in\Upsilon \mid y\in[\tilde{x}]\} \cup \{[\tilde{z}]\in\kappa \mid y\in[\tilde{z}]\}
	\end{align*}
\end{definition}
\bigskip

The last base, $\Pi$, is the approximated order of program points, where we denote $p\sqsubseteq p'\in\Pi$ if the program point $p$ is an earlier or the same program point as $p'$.
The order of program points is also transitive, i.e., if $p\sqsubseteq p'\in\Pi$ and $p'\sqsubseteq p''\in\Pi$ then $p\sqsubseteq p''\in\Pi$.


\subsection{Judgement}
The type judgement is defined as:
$$\Gamma,\Upsilon,\Pi\vdash t: T$$

In the $VAR$ rule we make use of the union between a type and a base type, which we define as follows:

\begin{definition}{[Union of types]}
	Let $T$ be an arbitrary type and $(\delta,\kappa)$ be a base type, then the unification of these two types is as follows:
	\begin{equation*}
		T\cup(\delta,\kappa)=\left\{\begin{matrix}
			\mbox{If } \; T=(\delta',\kappa') & \mbox{then} \; (\delta'\cup\delta,\kappa'\cup\kappa)\\
			\mbox{else if } \; T=x:T_1\rightarrow T_2 & \mbox{then} \; x:T_1\rightarrow (T_2\cup(\delta,\kappa))
		\end{matrix}\right.
	\end{equation*}
\end{definition}

\begin{landscape}
\subsection{Example Abs}
Given the following program
$$(\lambda x.(x\;z))(\lambda y.y)$$
Where $z$ is bound to a constant integer and its type is thus $(\emptyset,\emptyset)$.

To type this program, we first construct the tree according to the rules:

\begin{figure}[H]
	\setlength\tabcolsep{8pt}
		\begin{tabular}{l}
		\inference[$APP$]
		{
			\inference[$ABS$]
			{
				\inference[$APP$]
				{
					\inference[$VAR$]{}{\Gamma';\Pi\vdash x^{p}:\alpha:T'\rightarrow T''}
					\;\;\;
					\inference[$VAR$]{}{\Gamma';\Pi\vdash z^{p'}:(\{z^{p'}\},\emptyset)}
				}
				{\Gamma,x:T_1;\Pi\vdash [x^{p}\;z^{p'}]^{p''}:T''\{\alpha / (\{z^{p'}\},\emptyset)\}}
			}
			{\Gamma;\Pi\vdash [\lambda\; x.([x^{p}\;z^{p'}]^{p''})]^{p_3}:x:T_1\rightarrow T''\{\alpha / (\{z^{p'}\},\emptyset)\}}
			\;\;
			\inference[$ABS$]
			{
					\inference[$VAR$]{}{\Gamma,y:T;\Pi\vdash y^{p_3}:T}
			}
			{\Gamma';\Pi\vdash \lambda y.[y^{p_3}]^{p_4})]^{p_5}:y:T\rightarrow T}
		}
		{\Gamma;\Pi\vdash [[\lambda x.[x^{p}\;z^{p'}]^{p''}]^{p_3}\;[\lambda y.y^{p_4}]^{p_5}]^{p_6}:T''\{\alpha / (\{z^{p'}\},\emptyset)\}\{x / T\rightarrow T\}}\\
		Where $T_1=\alpha:T'\rightarrow T''$
	\end{tabular}

	\label{fig:AbsExample}
\end{figure}

After that, we can solve the intermediate types, since we now know the righthand type of the outermost application, and we will get the following:

\begin{figure}[H]
	\setlength\tabcolsep{8pt}
	\begin{tabular}{l}
		\inference[$APP$]
		{
			\inference[$ABS$]
			{
				\inference[$APP$]
				{
					\inference[$VAR$]{}{\Gamma';\Upsilon;\Pi\vdash x^{p}:\alpha:T\rightarrow T}
					\;\;\;
					\inference[$VAR$]{}{\Gamma';\Upsilon;\Pi\vdash z^{p'}:(\{z^{p'}\},\emptyset)}
				}
				{\Gamma,x:T\rightarrow T;\Upsilon;\Pi\vdash [x^{p}\;z^{p'}]^{p''}:T\{\alpha / (\{z^{p'}\},\emptyset)\}}
			}
			{\Gamma;\Upsilon;\Pi\vdash [\lambda\; x.([x^{p}\;z^{p'}]^{p''})]^{p_3}:x:(T\rightarrow T)\rightarrow T\{\alpha / (\{z^{p'}\},\emptyset)\}}
			\;\;
			\inference[$ABS$]
			{
					\inference[$VAR$]{}{\Gamma,y:T;\Upsilon;\Pi\vdash y^{p_3}:T}
			}
			{\Gamma';\Upsilon;\Pi\vdash \lambda y.[y^{p_3}]^{p_4})]^{p_5}:y:T\rightarrow T}
		}
		{\Gamma;\Upsilon;\Pi\vdash [[\lambda x.[x^{p}\;z^{p'}]^{p''}]^{p_3}\;[\lambda y.y^{p_4}]^{p_5}]^{p_6}:T\{\alpha / (\{z^{p'}\},\emptyset)\}\{x / T\rightarrow T\}}\\
	\end{tabular}
	\label{fig:Simple}
\end{figure}

Lastly, we can solve the last part, to get the prober type:

\begin{figure}[H]
	\setlength\tabcolsep{8pt}
	\begin{tabular}{l}
		\inference[$APP$]
		{
			\inference[$ABS$]
			{
				\inference[$APP$]
				{
					\inference[$VAR$]{}{\Gamma';\Upsilon;\Pi\vdash x^{p}:\alpha:T\rightarrow T}
					\;\;\;
					\inference[$VAR$]{}{\Gamma';\Upsilon;\Pi\vdash z^{p'}:(\{z^{p'}\},\emptyset)}
				}
				{\Gamma,x:T\rightarrow T;\Upsilon;\Pi\vdash [x^{p}\;z^{p'}]^{p''}:T\{\alpha / (\{z^{p'}\},\emptyset)\}}
			}
			{\Gamma;\Upsilon;\Pi\vdash [\lambda\; x.([x^{p}\;z^{p'}]^{p''})]^{p_3}:x:(T\rightarrow T)\rightarrow T\{\alpha / (\{z^{p'}\},\emptyset)\}}
			\;\;
			\inference[$ABS$]
			{
					\inference[$VAR$]{}{\Gamma,y:T;\Upsilon;\Pi\vdash y^{p_3}:T}
			}
			{\Gamma';\Upsilon;\Pi\vdash \lambda y.[y^{p_3}]^{p_4})]^{p_5}:y:T\rightarrow T}
		}
		{\Gamma;\Upsilon;\Pi\vdash [[\lambda x.[x^{p}\;z^{p'}]^{p''}]^{p_3}\;[\lambda y.y^{p_4}]^{p_5}]^{p_6}:(\{z^{p'}\},\emptyset)}\\
	\end{tabular}
	\label{fig:Simple}
\end{figure}
\subsection{Example case}
In the following example, the initial $\Gamma$ and $\Upsilon$ is given by:
\begin{align*}
	\Gamma &= [\nu x\mapsto(\emptyset,\emptyset),x\mapsto(\emptyset,\{\{\nu x\}\}),\nu y\mapsto(\emptyset,\emptyset),y\mapsto(\emptyset,\{\{\nu y\}\})]\\
	\Upsilon &=\{\{\nu x, x\},\{\nu y, y\}\}
\end{align*}
\begin{figure}[H]
	\setlength\tabcolsep{8pt}
	\begin{tabular}{l}
	\inference[$CASE$]
	{
		\inference[$Cons$]{}{\Gamma;\Upsilon;\Pi\vdash 1:(\emptyset,\emptyset) }
		\;\;
		\inference[$Pattern$]
		{
			\inference[$Var$]{} {\Gamma;\Upsilon;\Pi\vdash x:(\{x\},\{\{\nu x\}\})}
		}
		{\Gamma;\Upsilon;\Pi\vdash [1\;(x)]:(\{x\},\{\{\nu x\}\}) }
		&\\
		\inference[$Pattern$]
		{
			\inference[$Var$]{} {\Gamma;\Upsilon;\Pi\vdash y:(\{y\},\{\{\nu y\}\})}
		}
		{\Gamma;\Upsilon;\Pi\vdash [2\;(y)]:(\{y\},\{\{\nu y\}\}) }
		\;\;
		\inference[$Pattern$]
		{
			\inference[$Ref$]
			{
				\inference[$Cons$]{}{\Gamma;\Upsilon;\Pi\vdash 10:(\emptyset,\emptyset) }
			}
			{\Gamma,\nu z:(\emptyset,\emptyset);\Upsilon,\{\{\nu z\}\};\Pi\vdash [ref\;(10)]:(\emptyset,\{\{\nu z\}\})}
		}
		{\Gamma;\Upsilon;\Pi\vdash [3\;(ref(10))]:(\emptyset,\{\{\nu z\}\}) }
	}
{\Gamma;\Upsilon;\Pi\vdash [case\;(1)\;\{(1\;(x))\;(2\;(y))\;(3\;(ref\;(1)))\}]:(\{x,y\},\{\{\nu x\}, \{\nu y\}, \{\nu z\}\})}\\
\end{tabular}

	\label{fig:SimpleCase}
\end{figure}
This example will be used in the two following, and is refered as $caseEx$ with the type:
$$T=(\{x,y\},\{\{\nu x\}, \{\nu y\}, \{\nu z\}\})$$
\newpage

\subsection{Example let-case 1}
\begin{figure}[H]
	\setlength\tabcolsep{8pt}
	\begin{tabular}{l}
	\inference[$let$]
	{
		\inference[$Case$]{}
		{\Gamma;\Pi\vdash [caseEx]:T}
		\;\;
		\inference[$Ref\;read$]
		{
			\inference[$Var$]{}{\Gamma;\Pi\vdash z:T\cup(\{z\},\emptyset)}
		}
		{\Gamma,z:T;\Pi\vdash [!z]:T'}
	}
{\Gamma;\Pi\vdash [\mbox{let}\;z\;(caseEx)\;(!z)]:T'}\\
Where $T'=(\{x,y,z,\nu x,\nu y,\nu z\},\emptyset)$
\end{tabular}

	\label{fig:LetCase1}
\end{figure}

\subsection{Example let-case 2}
\begin{figure}[H]
	\setlength\tabcolsep{8pt}
	\begin{tabular}{l}
	\inference[$let$]
	{
		\inference[$Case$]{}
		{\Gamma;\Upsilon;\Pi\vdash [caseEx]:T}
		\;\;
		\inference[$Ref\;read$]
		{
			\inference[$Var$]{}{\Gamma;\Upsilon;\Pi\vdash x:(\{x\},\{\nu x, x\})}
		}
		{\Gamma,z:T;\Upsilon;\Pi\vdash [!x]:(\{x,\nu x\},\emptyset)}
	}
{\Gamma;\Upsilon;\Pi\vdash [\mbox{let}\;z\;(caseEx)\;(!x)]:T'}\\
Where $T'=(\{x,\nu x\},\emptyset)$
\end{tabular}

	\label{fig:LetCase1}
\end{figure}
\end{landscape}

\end{document}
